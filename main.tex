\documentclass[a4paper,twoside,12pt]{book}

% podstawowe pakiety
\usepackage[T1]{fontenc} % dzielenie wyrazów (word-break) na stronie
\usepackage{lmodern} % załadowanie czcionki latex modern
\usepackage[british,polish]{babel} % języki dokumentu
\usepackage{indentfirst} % wymusza wcięcie pierwszego akapitu po nagłówku (z defaultu latex pomija)
\usepackage{xurl} % automatyczne łamanie długich URLów
\usepackage{float} % pozycjonowanie obiektów
\usepackage{tabularx} % automatycznie rozciągana kolumna w tabeli
\usepackage[final,nopatch=footnote]{microtype} % poprawienie typografii (kerning, wysunięcie interpunkcji)
\usepackage[hidelinks]{hyperref} % linki bez kolorowania/ramek
\usepackage{graphicx} % umieszczanie grafik w dokumencie
\usepackage{booktabs} % do tworzenia tabel
\usepackage{tikz} % do tworzenia grafiki wektorowej w środku latexa
\usepackage{enumitem} % niestandardowy wygląd list
\usepackage{csquotes} % lepsze cudzysłowy
\usepackage{ifmtarg} % sprawdzenie, czy argument makra jest pusty
\usepackage{silence} % do wyłączania warningów
\usepackage{svg} % do generowania grafik SVG
\usepackage{longtable} % tabela używana do spisu skrótów i symboli
\usepackage[explicit]{titlesec} % niestandardowe style nagłówków (chapter, section, subsection)
\usepackage{fontspec} % do zarządzania czcionkami (nie działa z PDFLaTeX!, tylko xelatex/lualatex)
\usepackage{geometry} % rozstaw marginesów
\usepackage{setspace} % kontrola odstępu między wierszami
\usepackage{subcaption} % do stackowania obrazów w jednym bloku
\usepackage{caption} % podpisy pod rysunkami i tabelami
\usepackage{fancyhdr} % niestandardowe nagłówki (chapter, section, subsection)
\usepackage[page]{appendix} % dodatki
\usepackage[backend=biber,style=numeric-comp,sorting=nyt,sortcites=true,maxnames=9]{biblatex} % bibliografia
\usepackage{color} % kolory
\usepackage{xcolor} % ponownie kolory
\usepackage{textcomp} % dodatkowe symbole (waluty, jednostki miar itp.)
\usepackage{listings} % do listingów (kod źródłowy)


% czcionka calibri do strony tytułowej
\newfontfamily{\titlepagefonttest}{calibri}[Path=font/,Extension=.ttf,UprightFont=*-regular,BoldFont=*-bold]

\geometry{inner=2.5cm,outer=2.5cm,top=2.5cm,bottom=2.5cm} % geometria strony
\setstretch{1.4} % standardowa przestrzeń miedzy liniami tekstu 1.5
\frenchspacing % brak przerwy po kropkach
\setlength{\parskip}{5pt} % odstęp między akapitami

\setlist{itemsep=1pt} % przerwa między elementami listy
\setlist[itemize,2]{label=$\circ$} % ustaw pustą kropkę jako znak początku 2 poziomu listy

\babelprovide[transforms=oneletter.nobreak]{polish}
\babelprovide[justification=unhyphenated]{polish}
\hypersetup{hidelinks,breaklinks=true}

% nagłówki
\captionsetup{font=small,labelfont=bf,textfont=it,singlelinecheck=false} % podpisy elementów pływających
\setlength{\headheight}{14.5pt}

% ignoruj justowanie dla nagłówków
\titleformat{name=\chapter,numberless}[block]{\normalfont\huge\bfseries\raggedright}{}{0pt}{#1~}
\titleformat{name=\chapter}[display]{\normalfont\huge\bfseries\raggedright}{\chaptertitlename\ \thechapter}{20pt}{#1~}
\titleformat{name=\section}[hang]{\normalfont\Large\bfseries\raggedright}{\thesection}{20pt}{#1~}
\titleformat{name=\subsection}[hang]{\normalfont\large\bfseries\raggedright}{\thesubsection}{20pt}{#1~}

% obsługa dodatków
\renewcommand{\appendixtocname}{Dodatki}
\renewcommand{\appendixpagename}{Dodatki}
\renewcommand{\appendixname}{Dodatek}

% bibliografia
\addbibresource{biblio/biblio.bib}

\newcommand{\Type}{Projekt inżynierski}

\newcommand{\LeftId}{Nr albumu}
\newcommand{\LeftProgram}{Kierunek}
\newcommand{\LeftSpecialisation}{Specjalność}
\newcommand{\LeftSUPERVISOR}{PROWADZĄCY PRACĘ}
\newcommand{\LeftDEPARTMENT}{KATEDRA}

\newcommand{\FirstNameAuthor}{Miłosz}
\newcommand{\SurnameAuthor}{Gilga}
\newcommand{\IdAuthor}{297931}

\newcommand{\FirstNameSupervisor}{Marian}
\newcommand{\LastNameSupervisor}{Hyla}

\newcommand{\Title}{Komunikator internetowy oparty o technologie Websockets i Apache Kafka}
\newcommand{\TitleAlt}{Instant messenger system based on Websockets and Apache Kafka technologies}

\newcommand{\Program}{Informatyka}
\newcommand{\Specialisation}{Informatyka w systemach elektrycznych}
\newcommand{\Faculty}{Wydział Elektryczny}
\newcommand{\Departament}{Energoelektroniki, napędu elektrycznego i robotyki}

\newcommand{\Polsl}{Politechnika Śląska}
\newcommand{\Logo}{image/polsl_logo.pdf}

\newcommand{\City}{Gliwice}
\newcommand{\Year}{2024}

% kowariancje
\newcommand{\Author}{\FirstNameAuthor\ \MakeUppercase{\SurnameAuthor}}
\newcommand{\Supervisor}{dr inż. \FirstNameSupervisor\ \MakeUppercase{\LastNameSupervisor}}
 % import głównych danych dokumentu
\newcommand{\imagepath}{image/} % ścieżka do obrazów ogólnych (SVG)

% bezpieczny import obrazków svg (jeśli nie ma, teskt alternatywny)
\newcommand{\safeincludesvg}[2]{
  \IfFileExists{\imagepath#1.svg}{
    \includesvg[width=#2\textwidth]{\imagepath#1.svg}
  }{\textbf{[Obrazek nieznaleziony]}}
}

% bezpieczny import obrazków (jeśli nie ma, teskt alternatywny)
\newcommand{\safeincludepng}[2]{
  \IfFileExists{\imagepath#1.png}{
    \includegraphics[width=#2\textwidth]{\imagepath#1.png}
  }{\textbf{[Obrazek nieznaleziony]}}
}

% helper do referencji, #1 - prefix (rys.), #2 - opcjonalny (tekst po numerze np. a, b, c), #3 - nazwa referencji
\newcommand{\refhelper}[3]{#1\ref{#3}\ifx#2\empty\else#2\fi}
\newcommand{\imgref}[2][]{\refhelper{rys. }{#1}{#2}} % odwołanie do rysunku: (rys. X.Xa)
\newcommand{\lisref}[2][]{\refhelper{}{#1}{#2}} % odwołanie do listingu: (rys. X.Xa)

\newcommand{\english}[1]{{\selectlanguage{british}\emph{#1}}} % do obcego języka

% chapter, section, subsection - razem z labelem (możliwe odwołanie w tekście)
\newcommand{\makechapter}[2]{\chapter{#1} \label{#2}}
\newcommand{\makesection}[2]{\section{#1} \label{#2}}
\newcommand{\makesubsection}[2]{\subsection{#1} \label{#2}}
 % niestandardowe dyrektywy

% nagłówek i stopka
\pagestyle{fancy}
\fancyhf{}
\fancyhead[LO]{\nouppercase{\it\rightmark}}
\fancyhead[RE]{\nouppercase{\it\leftmark}}
\fancyhead[LE,RO]{\it\thepage}

% strony z nagłówkiem (nazwa rozdziału, autor) oraz stopką
\fancypagestyle{headerAndFooter}{
  \fancyhf{}
  \fancyhead[LE]{\nouppercase{\Author}}
  \fancyhead[RO]{\nouppercase{\leftmark}}
  \fancyfoot[CE, CO]{\thepage}
}

\definecolor{mediumgray}{rgb}{0.3, 0.4, 0.4}
\definecolor{mediumblue}{rgb}{0.0, 0.0, 0.8}
\definecolor{forestgreen}{rgb}{0.13, 0.55, 0.13}

\lstset{
  numbers=left, % numery linii po lewej stronie
  numbersep=10pt, % odstęp numerów od kodu: 10 punktów
  numberstyle=\color{black}, % numery linii: czarne (rozmiar z basicstyle)
  basicstyle=\ttfamily\footnotesize, % podstawowy styl: czcionka maszynowa, rozmiar \footnotesize
  breaklines=false, % długie linie NIE będą automatycznie łamane (nadpisuje wcześniejsze 'true')
  breakatwhitespace=false, % jeśli breaklines=true, łamie linie w dowolnym miejscu (nie tylko na spacjach)
  fontadjust=true, % uspójnienie wielkości czcionki
  showstringspaces=false, % spacje wewnątrz ciągów znaków (stringów) wyświetlane normalnie
  showspaces=false, % spacje w kodzie (poza stringami) wyświetlane normalnie
  showtabs=false, % znaki tabulacji konwertowane na spacje (zgodnie z tabsize), nieoznaczone
  showlines=true, % w pewnych sytuacjach może wizualizować "puste" lub kontynuowane linie (efekt często subtelny)
  xleftmargin=20pt,  % lewy margines/wcięcie całego bloku listingu: 20 punktów
  captionpos=b % pozycja podpisu (dół)
  columns=fullflexible, % elastyczne dopasowanie szerokości znaków w kolumnach
  commentstyle=\color{mediumgray}\upshape, % komentarze: kolor 'mediumgray', krój prosty
  identifierstyle=\color{black}, % identyfikatory (np. nazwy zmiennych): kolor czarny
  stringstyle=\color{forestgreen}, % ciągi znaków (stringi): kolor 'forestgreen'
  keywordstyle=\color{mediumblue}, % słowa kluczowe: kolor 'mediumblue'
  extendedchars=true, % umożliwia poprawne wyświetlanie znaków spoza ASCII
  fontadjust=true, % próbuje dostosować optycznie rozmiary czcionek między różnymi stylami
  keepspaces=true, % zachowuje wszystkie spacje w kodzie (w tym wielokrotne)
  upquote=true, % zapewnia poprawne wyświetlanie prostych apostrofów (') i cudzysłowów (")
  tabsize=2, % szerokość znaku tabulacji: 2 spacje
  title=\lstname, % używa nazwy pliku jako tytułu dla \lstinputlisting
  rulecolor=\color{black} % kolor linii ramki (jeśli ramka jest włączona, np. frame=single)
}

\lstdefinestyle{JSES6Base}{
  keywords={
      break, continue, delete, else, for, function, if, in, new, return, this, typeof, var, void, while, with,
      false, null, true, boolean, number, undefined, Array, Boolean, Date, Math, Number, String, Object,
      eval, parseInt, parseFloat, escape, unescape, await, async, case, catch, class, const, default, do, enum, export,
      extends, finally, from, implements, import, instanceof, let, static, super, switch, throw, try
    },
  sensitive,
  morecomment=[s]{/*}{*/},
  morecomment=[l]//,
  morecomment=[s]{/**}{*/},
  morestring=[b]',
  morestring=[b]"
}

% wrapper dla dokumentu, możliwość dodania elementów do 2 sekcji (rozdziały i dodatki)
\newcommand{\documentwrapper}[2]{
\newcounter{noNumberPages}

\begin{document}

\frontmatter % przed tekstem głównym, bez numeracji
\pagestyle{empty}{
  \newgeometry{top=1.5cm,bottom=2.5cm,left=3cm,right=2.5cm}

  \begingroup
  \titlepagefonttest

  \begin{center}
    \includegraphics[width=50mm]{\Logo}

    {\Large\bfseries\MakeUppercase{\Type}\par}

    \vfill \vfill

    {\large\Title\par}

    \vfill

    {\large\bfseries\Author\par}

    {\normalsize\bfseries \LeftId: \IdAuthor}

    \vfill

    {\large{\bfseries \LeftProgram:} \Program\par}

    {\large{\bfseries \LeftSpecialisation:} \Specialisation\par}

    \vfill \vfill \vfill \vfill \vfill \vfill \vfill

    {\large{\bfseries \LeftSUPERVISOR}\par}

    {\large{\bfseries \Supervisor}\par}

    {\large{\bfseries \LeftDEPARTMENT\ \Departament} \par}

    {\large{\bfseries \Faculty}\par}

    \vfill \vfill

    \vfill \vfill

    {\large\bfseries Gliwice \Year}
  \end{center}

  \endgroup
  \restoregeometry
}


\pagestyle{empty}
\cleardoublepage % nowa strona

\rmfamily\normalfont % reset fontów ze strony tytułowej

\subsubsection*{Tytuł pracy}
\Title

\subsubsection*{Streszczenie}
Projekt ma na celu przedstawienie możliwości wymiany danych między aplikacjami w środowisku rozproszonym przy użyciu
Apache Kafka oraz komunikacji z klientem z wykorzystaniem protokołu Websocket na przykładzie prostej aplikacji wymiany
wiadomości w czasie rzeczywistym. Wykonanie pracy polegało na stworzeniu dwóch osobnych warstw: wizualnej oraz
serwerowej. Architektura serwerowa została rozdzielona na kilka mikrousług odpowiadających za konkretne cele biznesowe
aplikacji (uwierzytelnianie, obsługa czatu, przetwarzanie plików itp.). Każdy mikroserwis stanowi osobny serwer
aplikacji z własną bazą danych. Komunikacja między mikroserwisami została zapewniona przez architekturę sterowaną
zdarzeniami i system Apache Kafka. Sama komunikacja z aplikacją klienta została zrealizowana przy użyciu interfejsu
REST, protokołu Websocket oraz zaimplementowanego serwera brzegowego. Wykonany system umożliwia kilka operacji. Są to
m.in. utworzenie konta, zarządzanie pokojami do rozmów, kanałów tekstowych oraz wysyłanie wiadomości z opcjonalnymi
załącznikami w formie plików. Stworzony system mimo swojej zewnętrznej prostoty zawiera wiele zaawansowanych konceptów z
zakresu architektury mikrousługowej. Rozwiązują one problemy aplikacji projektowanych na systemy rozproszone które nie
występują (lub występują w nieznacznym stopniu) w aplikacjach monolitycznych.

\subsubsection*{Słowa kluczowe}
system czatu, mikroserwisy, architektura sterowana zdarzeniami, apache kafka, websockety, apache cassandra, kontenery
docker

\newpage

\begin{otherlanguage}{british}

  \subsubsection*{Thesis title}
  \TitleAlt

  \subsubsection*{Abstract}
  The following project was designed to demonstrate the possibility of data exchange between applications in distributed
  environments using the Apache Kafka platform and communication with client using Websocket protocol on the example of
  a simple real-time messaging application. Main goal of this project involved creating two separate layers: visual
  layer and server layer. The server architecture was divided into several microservices responsible for specific
  business purposes in the application (authentication, chat, file processing, etc.). Each microservice is a separate
  application server with its own database. Communication between microservices was provided by the event-driven
  architecture and an Apache Kafka software. Communication between client application and server layer was implemented
  using a REST interface, Websocket protocol and an API gateway server. This system allows us to perform several
  operations. These include creating an account, managing chat rooms, text channels and sending messages with
  attachments. Despite its external simplicity, the created system incorporates many advanced concepts in the field of
  microservices architecture. These concepts address issues in applications designed for distributed systems that are
  not present (or are present to a lesser extent) in monolithic applications.

  \subsubsection*{Keywords}
  instant messenger system, microservices, event-driven architecture, apache kafka, websocket, apache cassandra, docker
  containers

\end{otherlanguage}


% spis treści (pusty styl strony bez nagłówka i stopki)
\addtocontents{toc}{\protect\thispagestyle{empty}}
\tableofcontents

\setcounter{noNumberPages}{\value{page}}

% główna część dokumentu
\mainmatter

\pagestyle{empty}
\cleardoublepage % nowa strona

\pagestyle{headerAndFooter} % strony z nagłówkiem i stopką

#1 % tutaj rozdziały

% końcowa część dokumentu
\backmatter

\printbibliography
\addcontentsline{toc}{chapter}{Bibliografia} % dodanie bibliografii do spisu treści

% dodatki
\begin{appendices}
  #2 % tutaj dodatki

  \listoffigures
  \addcontentsline{toc}{chapter}{Spis rysunków}
\end{appendices}

\end{document}
}


\widowpenalty=10000 % zapobiega występowaniu ostatnich linii akapitu na początku nowej strony
\clubpenalty=10000 % zapobiega występowaniu pierwszych linii akapitu na początku nowej strony
\vbadness=99999 % zapobiega ostrzeżeniom o niepełnych liniach

\documentwrapper{
  % rozdziały
  \makechapter{Wstęp}{chap:introduction}

Człowiek z definicji jest istotą społeczną, toteż potrzeba komunikacji jest u niego dość silna. Od wieków wymyślano
różne sposoby i techniki komunikacji na odległość. Początkowo komunikacja z wykorzystaniem listów była dość powolna i
mało bezpieczna. Wraz z rozwojem techniki, wynalezieniu telegrafu oraz stopniowemu powiększaniu się infrastruktury
telekomunikacyjnej wymiana wiadomości stawała się szybsza, przyjemniejsza oraz bezpieczniejsza. Rozwój Internetu w
latach 90 dwudziestego wieku wprowadził nowe możliwości z zakresu komunikacji dzięki zwiększeniu prędkości transmisji i
możliwości multimedialnych. Wówczas powstał jeden z najpopularniejszych polskich komunikatorów: Gadu-gadu który (na
dzień tworzenia tej pracy) wciąż istnieje na rynku \cite{bib:mszutiak}. W świecie, jaki jest obecnie znany, komunikatory
internetowe zajmują mocne miejsce na rynku aplikacji użytkowych.

Wraz z pojawieniem się pandemii Covid-19 w 2019 roku zapotrzebowanie rynkowe na komunikatory znacząco się zwiększyło. Z
uwagi na obostrzenia wynikające ze stanu epidemiologicznego, wszelkiego rodzaju formy wymiany informacji, spotkań oraz
rozmowy z najbliższymi odbywały się z użyciem technik komunikacji na odległość. Niektóre z aplikacji oferujących wyżej
wymienione techniki nie były gotowe na tak duże obciążenie generowane przez szkoły, firmy lub inne instytucje
korzystające w tym samym czasie z ich usług. Skutkowało to utrudnieniem w komunikacji, bądź nierzadko przerwami w ich
działaniu, znacząco utrudniając komunikację.

Motywacją do stworzenia projektu o podanej tematyce było opracowanie prostego systemu do wymiany wiadomości, lecz
możliwie jak najbardziej skalowanego, bezpiecznego i przygotowanego do stosunkowo łatwego wdrożenia w środowiskach
rozproszonych na wielu serwerach. Projekt taki może stanowić bazę do bardziej rozbudowanego systemu bądź zapewniać
prosty system do wymiany wiadomości w czasie rzeczywistym. W celu stworzenia projektu wykorzystano szereg nowoczesnych
technologii stworzonych i aktywnie wykorzystywanych przez gigantów technologicznych takich jak \textit{Netflix} i
\textit{LinkedIn}.

\makesection{Cel i zakres projektu}{sec:project-purpose}

Celem projektu było opracowanie skalowalnego systemu wymiany wiadomości w czasie rzeczywistym wykorzystującego
architekturę mikroserwisów. Komunikacja pomiędzy usługami warstwy serwera została zapewniona przy użyciu oprogramowania
\textit{Apache Kafka} implementującego model komunikacji sterowanej zdarzeniami. Z kolei do wzajemnej dwukierunkowej
wymiany danych w czasie rzeczywistym między klientem a serwerem skorzystano z protokołów \textit{Websocket} i
\textit{STOMP}.

\makesection{Założenia funkcjonalne}{sec:functional-assumptions}

W podstawowy swoich założeniach system ma umożliwiać wysyłanie wiadomości w obrębie kanału tekstowego w tym wiadomości w
formie załączonych plików (tekstowych, graficznych, audio, wideo itp.). Kanał taki znajduje się na serwerze zwanym dalej
\textit{Sferą}, na którą składają się wyżej wymienione kanały tekstowe i użytkownicy. Każda Sfera posiada jednego
administratora zdolnego do tworzenia, modyfikowania oraz usuwania kanałów tekstowych. Dołączanie użytkowników do
\textit{Sfery} umożliwione jest dzięki linkowi z zaproszeniem generowanym przez administratora z możliwością określenia
czasu jego wygaśnięcia. Administrator \textit{Sfery} ma możliwość wyrzucenia oraz zablokowania użytkownika. Ponadto
aplikacja umożliwia podstawowe elementy uwierzytelniania i autoryzacji tj. logowanie, rejestrację, reset hasła przy
pomocy adresu email oraz logowanie społecznościowe (z wykorzystaniem konta \textit{Google} lub \textit{Facebook}). Co
więcej, konto w systemie można zabezpieczyć poprzez weryfikację wieloetapową z użyciem jednorazowych kodów aplikacji
\textit{Microsoft Authenticator} lub \textit{Google Authenticator}.

  \makechapter{Wykorzystane oprogramowanie}{chap:used-software}

\makesection{Wykorzystane oprogramowanie do rozwoju warstwy klienta}{sec:client-software}

W celu stworzenia warstwy klienta składającego się na główny projekt \textit{SPA} napisany we \textit{frameworku}
\textit{Angular} oraz podprojekt zawierający style biblioteki \textit{Tailwind} wraz z dodatkową konfiguracją
\textit{Webpack} skorzystano z programu \textit{Visual Studio Code}. Jest to lekki edytor kodu źródłowego dostępnego na
systemy \textit{Windows}, \textit{MacOS} oraz \textit{Linux}. Możliwe jest instalowanie do niego rozszerzeń
integrujących elementy innych języków oraz frameworków (takie jak kompilatory czy walidatory składni) \cite{bib:vscode}.
W powyższym projekcie skorzystano głównie z rozszerzeń ekosystemu \textit{Angular} oraz \textit{Tailwind}
umożliwiających sprawdzanie poprawności kodu przed jego interpretacją co znacznie usprawniło i przyspieszyło rozwój
systemu.

\makesection{Wykorzystane oprogramowanie do rozwoju warstwy serwera}{sec:server-software}

W celu stworzenia warstwy serwerowej wykorzystano programy firmy \textit{JetBrains}. Wśród nich znajduje się flagowy
produkt tej firmy — \textit{Intellij Idea}. Jest to program typu \textit{IDE} stosowany do pisania, testowania oraz
kompilowania programów uruchamianych na \textit{JVM} (napisanych w \textit{Javie}, ale również w innych językach takich
jak \textit{Groovy} czy \textit{Kotlin}). Podobnie jak \textit{Visual Studio Code} umożliwia instalowanie wtyczek
dodających nowe funkcje \cite{bib:intellij-idea}. W projekcie skorzystano ze wtyczki \textit{Apache Kafka} pozwalającej
na obserwację napływających danych do strumieni co znacznie usprawniło walidację poprawnego przepływu danych między
serwisami.

Dodatkowo skorzystano z programu \textit{DataGrip} (również od firmy \textit{JetBrains}) będącym kompleksowym
środowiskiem bazodanowym. Aplikacja ta umożliwia podpięcie wielu różnych baz danych (różnych typów o różnych
specyfikacjach) i sprawnie nimi zarządzanie \cite{bib:intellij-datagrip}. Funkcja ta była dużym atutem w trakcie rozwoju
aplikacji ze względu na wykorzystaną architekturę, która narzuca stosowanie odrębnej bazy danych dla każdego
mikroserwisu.

\makesection{System kontroli wersji \textit{GIT}}{sec:git}

W rozwoju całego systemu korzystano z systemu kontroli wersji \textit{GIT}. System kontroli wersji to oprogramowanie
zapisujące historię edycji plików z kodem źródłowym. Sam system \textit{GIT} został stworzony przez Linusa Torvaldsa,
twórcę jądra systemu operacyjnego \textit{Linux} \cite{bib:git-history}. Dodatkowo system taki jest rozproszony, dzięki
czemu umożliwia pracę na wielu różnych gałęziach nad różnymi funkcjami systemu w tym samym czasie
\cite{bib:git-overview}. Stał się on bardzo przydatny podczas dodawania nowych funkcjonalności do dosyć złożonego
architektoniczne projektu przy pracy z wykorzystaniem dwóch komputerów.

\makesection{System konteneryzacji \textit{Docker}}{sec:docker}

W celu ułatwienia przyszłego wdrożenia systemu w środowiskach produkcyjnych skorzystano z systemu konteneryzacji
\textit{Docker}. Sam system \textit{Docker} powstał jako szybsza i wydajniejsza alternatywa dla tradycyjnych systemów
wirtualizacji. Główne dwa elementy w systemie \textit{Docker} to obraz i kontener. Obraz zawiera cały kod źródłowy i
wszystkie pliki niezbędne do uruchomienia aplikacji. Kontener zaś to odseparowane środowisko (tzw.
\textit{piaskownica}), które umożliwia uruchomienie wcześniej przygotowanego obrazu jako niezależny proces na komputerze
hosta \cite{bib:docker-overview}. Umożliwia to przede wszystkim uruchamianie wielu instancji tej samej aplikacji z taką
samą konfiguracją niezależnie od zastosowanej maszyny, co jest częstą praktyką w systemach rozproszonych opartych o
architekturę mikroserwisową w celu równoważenia obciążenia.

  \makechapter{Architektura systemu}{chap:system-architecture}

\makesection{Zarys ogólnej architektury systemu}{sec:architecture-overview}

System podzielony jest na dwie główne warstwy: warstwę klienta oraz serwera. Warstwa klienta napisana jest z
wykorzystaniem \textit{frameworka} \textit{Angular} w języku \textit{Typescript} oraz biblioteki \textit{Tailwind}, a
warstwa serwera z wykorzystaniem \textit{frameworka} \textit{Spring} oraz języka \textit{Java} w architekturze
rozproszonej (mikroserwisowej). Warstwy systemu komunikują się ze sobą przy pomocy protokołu \textit{HTTP} oraz
\textit{Websocket} w zależności od funkcjonalności. Sam sposób komunikacji przy pomocy protokołu \textit{HTTP} został
zrealizowany poprzez bezstanową specyfikację \textit{REST}.

\makesection{Infrastruktura warstwy klienta}{sec:client-infra}

Infrastruktura warstwy klienta (\imgref{fig:client-infra}) składa się z dwóch głównych elementów: aplikacji klienta
uruchamianej na serwerze \textit{HTTP} (\textit{Nginx}) oraz serwera typu \textit{S3} przechowującego statyczne pliki
(obrazy, czcionki, pliki tłumaczeń itp.). Sama aplikacja jest aplikacją webową, toteż komunikacja z nią odbywa się
poprzez przeglądarkę internetową z wykorzystaniem protokołu \textit{HTTP} oraz \textit{Websocket}. Zarówno aplikacja
klienta, jak i infrastruktura serwerowa korzystają z zasobów serwera \textit{S3} udostępnianych poprzez protokół
\textit{HTTP}. Komunikacja z warstwą serwerową odbywa się jedynie poprzez główną bramę \textit{API} z uwagi na
zastosowanie architektury rozproszonej.
%
\begin{figure}[H]
  \centering
  \safeincludesvg{3_2/0_1_client_infra}{0.9}
  \caption{Infrastruktura warstwy klienta.}
  \label{fig:client-infra}
\end{figure}

Przechowywanie zasobów statycznych na osobnym serwerze \textit{S3} niesie za sobą szereg korzyści. Jedną z nich jest
odciążenie głównego serwera aplikacji klienta poprzez zastosowanie sieci \textit{CDN}. Węzły takiej sieci rozlokowane w
wielu centrach danych na całym świecie znacząco skracają czas pobierania zasobów przez wielu użytkowników jednocześnie z
różnych części świata.

\makesubsection{Zastosowanie \textit{frameworka} \textit{Angular} jako systemu \textit{SPA}}{sub:client-infra-angular}

Warstwa klienta została zrealizowana z wykorzystaniem \textit{frameworka} \textit{Angular}. \textit{Framework} ten
stworzony został przez firmę \textit{Google} i jest jednocześnie następcą powstałego w 2010 roku \textit{AngularJS}.
Umożliwia on budowanie jednostronicowych aplikacji internetowych (\textit{SPA}), czyli takich, w których zapytania do
serwera wykonywane są w technologii \textit{AJAX} bez potrzeby każdorazowego przeładowania strony. W przeciwieństwie do
poprzednika korzysta on z języka \textit{Typescript} co znacznie ułatwia utrzymywanie i debugowanie dużych aplikacji z
uwagi na system typów. Sam \textit{framework} działa bazując na architekturze komponentowej, z widocznym wydzieleniem na
warstwę logiki biznesowej oraz widoku \cite{bib:ts-fain-2019}.

Aplikacja klienta została podzielona na kilka leniwie (ang. \english{lazy}) ładowanych modułów (tj. ładowanych, dopiero
gdy wybrana podstrona korzysta z kodu modułu — \imgref{fig:client-modules}). Ogranicza to ilość potrzebnych plików do
załadowania przy starcie aplikacji, co znacznie skraca czas jej uruchamiania. Same moduły zostały podzielone pod
względem funkcjonalnym na odpowiednio: \english{auth} (uwierzytelnianie), \english{client} (moduł aplikacji głównej),
\english{settings} (moduł ustawień aplikacji) oraz \english{shared} (wspólne komponenty dla wszystkich modułów).
%
\begin{figure}[H]
  \centering
  \safeincludesvg{3_2/1_1_client_modules}{0.7}
  \caption{Podział aplikacji klienta na moduły.}
  \label{fig:client-modules}
\end{figure}

\makesubsection{Rola biblioteki \textit{Webpack} oraz \textit{Babel} w warstwie klienta}{sub:client-infra-webpack}

Jak wspomniano we wcześniejszym podrozdziale, aplikacja została napisana z wykorzystaniem języka \textit{Typescript},
toteż w celu przetworzenia kodu \textit{Typescript} do zrozumiałego dla przeglądarek wysokowydajnego i zminifikowanego
kodu \textit{Javascript} wykorzystano niestandardowe konfiguracje bibliotek \textit{Webpack} oraz \textit{Babel}.
\textit{Webpack} odpowiada za połączenie wszystkich zasobów klienta (pliki \textit{Typescript}, obrazy oraz arkusze
styli). \textit{Typescript} natomiast odpowiada za transpilowanie\footnote{Transpilacja — konwersja kodu źródłowego w
  równoważnym języku programowania. W przeciwieństwie do kompilacji transpilacja działa na tym samym poziomie abstrakcji
  \cite{bib:transpiling}.} kodu dla standardów zrozumiałych przez starsze przeglądarki internetowe. Zdecydowano się na
niestandardowe konfiguracje z uwagi na osobny projekt przechowujący wszystkie style projektu napisane z wykorzystaniem
technologii \textit{Tailwind}. Takie rozwiązanie przyszłościowo ułatwi tworzenie klienta typu desktop wykorzystującego,
chociażby technologię \textit{Electron}. Na \imgref{fig:client-compile} pokazano zaimplementowany proces kompilowania i
transpilowania kodu \textit{Typescript} oraz łączenia dodatkowych zależności w końcową wersję produkcyjną możliwą do
uruchomienia z wykorzystaniem kontenera \textit{Typescript} i serwera \textit{HTTP} \textit{Nginx}.
%
\begin{figure}[H]
  \centering
  \safeincludesvg{3_2/2_1_client_compile}{0.8}
  \caption{Proces przetworzenia kodu źródłowego klienta na kod produkcyjny.}
  \label{fig:client-compile}
\end{figure}

\makesubsection{Zastosowanie wzorca reaktywnego w aplikacji klienta}{sub:client-reactive-pattern}

W aplikacji klienta w dużym stopniu korzystano ze wzorca reaktywnego zapewnianego przez bibliotekę \textit{RXJS}.
Biblioteka ta umożliwia tworzenie obserwowalnych strumieni, modyfikowane emitowanych wartości poprzez operatory oraz
reagowanie na zmiany wykorzystując mechanizm subskrypcji. Główne elementy tej biblioteki to klasy \verb|Subject|,
\verb|BehaviorSubject| oraz \verb|Observable| odpowiadające za tworzenie wyżej wymienionych strumieni. Programowanie
reaktywne zostało wykorzystane w aplikacji w miejscach, w których zachodzi potrzeba asynchronicznej wymiany danych np. z
zapytania \textit{HTTP} do widoku.

Na listingu \lisref{lis:rxjs} widoczna jest przykładowa implementacja pobierania szczegółów kanału tekstowego.
Każdorazowe pobranie uruchamiane jest przy wyemitowaniu wartości do \verb|_onChangeObserver$| przy aktualizacji lub
usunięciu kanału tekstowego lub zmianie ścieżki (inny kanał tekstowy). Dzięki takiemu podejściu dane w interfejsie
użytkownika są na bieżąco aktualizowane i spójne z danymi w bazie danych. Widoczny jest również szereg operatorów
modyfikujących strumień (takich jak \verb|map|, \verb|filter| czy \verb|switchMap|).
%
\begin{lstlisting}[style=JSES6Base,label={lis:rxjs},caption={Przykład zastosowania biblioteki \textit{RXJS}.}]
fetchTextChannelDetails$(
  route: ActivatedRoute
): Observable<TextChannelDetailsResDto> {
  return combineLatest([route.paramMap, this._onChangeObserver$]).pipe(
    tap(() => this.setFetching(true)),
    distinctUntilChanged(),
    map(([paramMap]) => Number(paramMap.get('textChannelId'))),
    filter(textChannelId => !!textChannelId),
    switchMap(textChannelId => {
      this._textChannelId$.next(textChannelId);
      return this._guildManagementHttpClientService
        .getTextChannelDetails$(textChannelId);
    }),
    tap(textChannelDetails => {
      this._textChannelDetails$.next(textChannelDetails);
      this.setFetching(false);
    }),
    catchError(err => {
      this.setFetching(false);
      return throwError(() => err);
    })
  );
}
\end{lstlisting}

\makesection{Architektura warstwy serwera}{sec:server-architecture}

Architektura serwera została napisana z wykorzystaniem skalowalnej infrastruktury mikroserwisowej przy użyciu
rozwiązania \textit{Spring Cloud Netflix}. Odróżnia się od klasycznej architektury typu monolit podziałem dużej
aplikacji na kilka mniejszych, działających w heterogenicznych środowiskach oraz możliwych do uruchomienia na różnych
zdecentralizowanych serwerach. Na \imgref{fig:server-infra} widoczna jest cała architektura z uwzględnieniem mikrousług
skierowanych na konkretne cele biznesowe, bazy danych oraz dodatkowe elementy systemu niezbędne do poprawnego jego
działania w architekturze rozproszonej. Wśród tych składowych można rozróżnić kilka mikrousług wykonujących konkretne
cele biznesowe:
%
\begin{itemize}
  \item \textbf{user service} — przechowywanie danych użytkowników aplikacji, obsługa uwierzytelniania i autoryzacji
        (logowanie, rejestracja) z wyłączeniem kont obsługiwanych przez zewnętrznych dostawców (\textit{OpenID}),
  \item \textbf{sphere service} — przechowywanie i zarządzanie Sferami oraz kanałami tekstowymi przez zarządców,
        dodawanie i usuwanie członków \textit{Sfer},
  \item \textbf{settings service} — przechowywanie i zarządzanie ustawieniami użytkowników w aplikacji (ustawienia
        języka, motywu itp.),
  \item \textbf{oauth2 client service} — autoryzacja i uwierzytelnianie użytkowników przez zewnętrznych dostawców —
        \textit{Google} oraz \textit{Facebook} z wykorzystaniem specyfikacji \textit{OAuth2} oraz \textit{OpenID},
  \item \textbf{notification service} — obsługa powiadomień w aplikacji w tym powiadomień w formie wiadomości email,
  \item \textbf{multimedia service} — obsługa plików w aplikacji w tym: generowanie obrazów, kompresja plików
        przesyłanych z czatu oraz zapis i odczyt tych plików na serwerze typu \textit{S3},
  \item \textbf{chat service} — obsługa czatu aplikacji (wysyłania oraz odbierania wiadomości w czasie rzeczywistym).
\end{itemize}

Oprócz wymienionych mikrousług znajdują się również dodatkowe usługi charakterystyczne dla architektury rozproszonej. Są
to odpowiednio:
%
\begin{itemize}
  \item \textbf{config server} — serwer przechowujący pliki konfiguracyjne wszystkich mikroserwisów oraz innych serwisów
        systemu,
  \item \textbf{vault} — serwer przechowujący wrażliwej informacje systemu (klucze dostępu do baz danych, poświadczenia
        oraz zestaw kluczy dla \textit{Apache Kafka}, itp.),
  \item \textbf{discovery server} — serwer rejestrujący nowe instancje mikroserwisów oraz propagujący ich stan (aktywna
        lub nieaktywna),
  \item \textbf{serwer brzegowy} — serwer pośredniczący składający się na bramę \textit{API} (ang. \english{API
          gateway}) oraz moduł równoważenia obciążenia (ang. \english{load balancer}),
  \item \textbf{kafka} — broker wiadomości używany do komunikacji między mikroserwisami infrastruktury przy
        wykorzystaniu architektury sterowanej zdarzeniami.
\end{itemize}
%
\begin{figure}[H]
  \centering
  \safeincludesvg{3_3/0_1_server_infra}{1}
  \caption{Infrastruktura warstwy serwera.}
  \label{fig:server-infra}
\end{figure}

\makesubsection{
  Zastosowanie frameworka \textit{Spring} oraz \textit{Spring Cloud} w środowisku \textit{JVM}
}{sub:server-spring-jvm}

Mikrousługi w systemie zostały napisane przy użyciu \textit{frameworka} \textit{Spring} i języka \textit{Java}.
\textit{Framework} ten powstał w 2002 roku i stale zyskiwał na popularności stając się najpopularniejszym rozwiązaniem
umożliwiającym tworzenie usług webowych w języku \textit{Java}. W projekcie skorzystano z automatycznych konfiguracji
zapewnianych przez \textit{Spring Boot} co znacznie przyspieszyło i zredukowało tzw. \english{boilerplate code}, czyli
powtarzalny, ale niezbędny kod do skonfigurowania i uruchomienia aplikacji. Dodatkowo skorzystano z wielu podprojektów
\textit{Spring}, tj. \textit{Spring Security} czy \textit{Spring Data JPA} zapewniające interfejsy umożliwiające
odpowiednio uwierzytelnianie i autoryzacje użytkowników oraz zapewniające dostęp do bazy danych
\cite{bib:spring-walls-2019}. Każdy mikroserwis w projekcie stworzony z wykorzystaniem \textit{frameworka}
\textit{Spring} działa w ramach kontenera \textit{Tomcat} będącego serwerem aplikacji \textit{Java}. Serwer aplikacji
działa w ramach kontenera \textit{Docker}, toteż możliwe jest uruchomienie kilku takich kontenerów bez wzajemnego wpływu
na siebie.

Dodatkowo skorzystano z rozwiązania \textit{Spring Cloud Netflix}, czyli projektu stanowiącego zestaw narzędzi do
tworzenia rozwiązań systemów rozproszonych \cite{bib:spring-cloud-netflix}. W projekcie wykorzystano przede wszystkim
serwer konfiguracji, serwer brzegowy oraz \textit{service discovery} (\textit{Eureka}). Jak widać na
\imgref{fig:server-infra} serwer konfiguracji (\textit{config}) komunikuje się z magazynem kluczy \textit{Vault} oraz
repozytorium \textit{Github}. Wynika to z faktu, że cała konfiguracja infrastruktury serwerowej przechowywana jest w
jednym repozytorium jako pojedyncze źródło prawdy i jest taka sama dla wielu instancji jednego mikroserwisu. Dodatkowo
wrażliwe dane, takie jak klucze dostępu do elementów infrastruktury serwera przechowywane są w magazynie \textit{Vault}
(\imgref{fig:vault}) na osobnym serwerze zabezpieczonym kilkoma tokenami dostępu (\imgref{fig:server-infra}).
%
\begin{figure}[H]
  \centering
  \safeincludepng{3_3/1_1_vault}{1}
  \caption{Magazyn kluczy Vault.}
  \label{fig:vault}
\end{figure}

Na \imgref{fig:server-infra} widoczna jest również usługa \textit{service discovery}. Jak wspomniano we wcześniejszym
podrozdziale umożliwia ona rejestracje nowych instancji mikroserwisów oraz dostarcza panel sterowania
(\imgref{fig:spring-eureka}) zapewniający podgląd statusu zarejestrowanych usług w systemie (odpowiednio \textit{UP} i
\textit{DOWN} dla aktywnej lub nieaktywnej usługi). Przy nazwie usługi dodawany jest dodatkowo losowy identyfikator w
celu odróżnienia dwóch instancji tej samej mikrousługi.
%
\begin{figure}[H]
  \centering
  \safeincludepng{3_3/1_2_spring_eureka}{1}
  \caption{Panel sterowania usługi \textit{service discovery} (\textit{Eureka}).}
  \label{fig:spring-eureka}
\end{figure}

\makesubsection{Bezstanowe uwierzytelnianie z wykorzystaniem specyfikacji \textit{JWT}}{sub:server-jwt}

Z uwagi na zastosowaną architekturę mikroserwisową uwierzytelnianie oraz autoryzacja użytkowników muszą być bezstanowe,
tj. serwer nie może przechowywać sesji o zalogowanych użytkownikach a uwierzytelniać ich przy każdym wykonanym żądaniu
wymagającym podania poświadczeń. W zrealizowanym systemie w tym celu wykorzystano specyfikację \textit{JWT}. Na
\imgref{fig:jwt-schema} przedstawione są kolejne etapy weryfikacji poświadczeń poprzez logowanie z użyciem hasła. Po
udanej weryfikacji tworzone są dwa tokeny: \textit{access} oraz \textit{refresh}, z których ten pierwszy jest ważny 5
minut a refresh przypada na czas sesji użytkownika. Po wylogowaniu \textit{access token} umieszczany jest w tabeli
\verb|blacklist_jwts|, aby niemożliwe było jego ponowne użycie, a \textit{refresh token} jest niszczony, aby zakończyć
sesję użytkownika. Dodatkowo sam token podpisywany jest tajnym kluczem, aby niemożliwym było jego podrobienie.
%
\begin{figure}[H]
  \centering
  \safeincludesvg{3_3/2_1_jwt_schema}{1}
  \caption{Diagram przedstawiający logowanie i wylogowanie przy użyciu \textit{JWT}.}
  \label{fig:jwt-schema}
\end{figure}

\makesubsection{Komunikacja pomiędzy elementami infrastruktury serwerowej}{sub:server-communication}

W standardowej architekturze mikroserwisowej komunikacja między serwisami odbywa się zazwyczaj z wykorzystaniem
standardowych zapytań \textit{HTTP}. Zdecydowano się odejść od tego podejścia na rzecz architektury sterowanej
zdarzeniami z wykorzystaniem \textit{Apache Kafka} z uwagi na większą przepustowość, a co za tym idzie większą szybkość
przesyłania komunikatów. W celach deweloperskich skorzystano z jednego \textit{brokera} (serwera \textit{Kafki}), lecz
sam system został przygotowany do pracy z większą ilością \textit{brokerów}. Do zarządzania pracą \textit{brokerów} w
klastrze\footnote{Klaster — zbiór komputerów lub maszyn wirtualnych w rozproszonej sieci serwerów działających
  równolegle \cite{bib:cluster-definition}.} \textit{Kafki} wykorzystano oprogramowanie \textit{Zookeeper} umożliwiające
konfigurację i synchronizację oraz równoważenie obciążenia między wieloma \textit{brokerami}. Każdy \textit{broker}
posiada tematy podzielone na partycje. Za zarządzanie, która wiadomość powinna zostać odczytana odpowiada wartość
przesunięcia (ang. \english{offset}). Jest ona każdorazowo inkrementowana, dzięki czemu niemożliwym jest nadpisanie
wiadomości. Do klastra może przyłączyć się wiele konsumentów oraz producentów \cite{bib:kafka-overview}. Producent
dodaje dane (wiadomość) do wybranego tematu, a konsument odbiera. Przedstawienie tego procesu widoczne jest na
\imgref{fig:kafka-async}.
%
\begin{figure}[H]
  \centering
  \safeincludesvg{3_3/3_1_kafka_async}{1}
  \caption{
    Schemat działania klastra \textit{Apache Kafka}. Opracowanie własne na podstawie \cite{bib:kafka-cluster-image}.
  }
  \label{fig:kafka-async}
\end{figure}

W omawianym systemie wykorzystano dwa modele przesyłania komunikatów przy użyciu \textit{Apache Kafka}:
%
\begin{itemize}
  \item \textbf{model asynchroniczny} — wykorzystany w komunikacji między serwisami a mikroserwisem odpowiadającym za
        wysyłanie wiadomości email z uwagi na brak wiadomości zwrotnej (\imgref{fig:kafka-async}),
  \item \textbf{model synchroniczny} — oparty o \english{Request Reply Enterprise Integration Pattern} wykorzystujący
        kanały asynchroniczne do zasymulowania modelu żądanie-odpowiedź (wykorzystany przede wszystkich w transakcjach
        rozproszonych).
\end{itemize}

Wspomniany model synchroniczny wykorzystuje temat nadawania wiadomości oraz $N$ tematów odpowiedzi, gdzie $N$ to liczba
mikroserwisów zainteresowanych odbiorem komunikatu odpowiedzi. Każdy serwis posiada własny identyfikator, na podstawie
którego tworzy temat odpowiedzi. Dzięki temu w przypadku wielu instancji tego samego serwisu \textit{Kafka} rozpozna, z
którego mikroserwisu nadano wiadomość i do odeśle do tego tematu odpowiedzi, który będzie miał ten sam identyfikator
\cite{bib:kafka-req-reply}. Przesyłanie identyfikatora odbywa się z wykorzystaniem nagłówka \verb|CORRELATION_ID|.
Procedura z wykorzystaniem modelu synchronicznego widoczna jest na \imgref{fig:kafka-sync}. Dodatkowo obecny jest moduł
równoważenia obciążenia zapewniany przez \textit{Zookeeper} w przypadku wykorzystania tych samych tematów na różnych
\textit{brokerach} \textit{Kafki}.
%
\begin{figure}[H]
  \centering
  \safeincludesvg{3_3/3_2_kafka_sync}{1}
  \caption{Schemat działania modelu synchronicznego \textit{Kafki}.}
  \label{fig:kafka-sync}
\end{figure}

\makesubsection{Komunikacja z klientem z wykorzystaniem serwera brzegowego}{sub:server-communication-edge-server}

Jak wspomniano na początku rozdziału, aplikacja klienta komunikuje się z serwerem w głównej mierze przy pomocy protokołu
\textit{HTTP} i specyfikacji \textit{REST}. Z uwagi na zastosowaną architekturę mikroserwisów (możliwość uruchamiania
wielu instancji tego samego mikroserwisu) klient musi wiedzieć, z którym mikroserwisem powinien się połączyć, aby
zrealizować żądanie. Przy jednej instancji mikroserwisu jest to stosunkowo proste, lecz przy wielu stanowi to problem
niejednoznaczności. Rozwiązano go przy pomocy zaimplementowanego serwera brzegowego (\imgref{fig:edge-server})
składającego się na główną bramę \textit{API} (ang. \english{API gateway}) oraz modułu równoważenia obciążenia (ang.
\english{load balancer}). Sama brama wystawia odpowiednie punkty końcowe \textit{REST} dla aplikacji klienta a moduł
równoważenia obciążenia odpowiada za właściwe trasowanie żądań do instancji mikroserwisów (w przypadku wielu instancji
tego samego mikroserwisu zgodnie z przyjętym algorytmem trasowania — w projekcie skorzystano z algorytmu \english{Round
  Robin} kierującego żądania sekwencyjnie do kolejnych instancji usługi). Specyfikacja taka rozwiązuje problem
niejednoznaczności żądań oraz wprowadza dodatkowe zabezpieczenie między klientem a samymi usługami, wystawiając jeden
wspólny interfejs \textit{REST} a tym samym ukrywając punkty końcowe usług biznesowych systemu.
%
\begin{figure}[H]
  \centering
  \safeincludesvg{3_3/4_1_edge_server}{1}
  \caption{Komunikacja z wykorzystaniem serwera brzegowego.}
  \label{fig:edge-server}
\end{figure}

\makesubsection{
  Dwukierunkowa wymiana danych w systemie czatu z wykorzystaniem protokołów \textit{Websocket} i \textit{STOMP}
}{sub:server-websocket-stomp}

Zgodnie z założeniami projektu w systemie czatu skorzystano z protokołu \textit{Websocket}. Protokół ten powstał w 2011
roku i umożliwia dwukierunkową wymianę informacji między serwerem a klientem. Idea jego działania opiera się na
utrzymywaniu otwartego kanału komunikacyjnego w przeciwieństwie do protokołu \textit{HTTP}, w którym po każdym
otrzymaniu odpowiedzi z żądania, połączenie jest zamykane \cite{bib:mdn-websockets}. Sam w sobie protokół
\textit{Websocket} jest najniższą warstwą abstrakcji, przez co integracja z innymi usługami (takimi jak np.
\textit{broker} komunikatów) jest skomplikowana. Z tego względu użyto rozwiązania \textit{STOMP} (ang. \english{Simple
  Text Oriented Messaging Protocol}). Sam \textit{STOMP} nie jest ściśle związany z protokołem \textit{Websocket}, ale
zapewnia wspólny sposób komunikacji w ramach różnych \textit{brokerów} komunikatów. Umożliwia on również dodawanie
nagłówków do połączenia \textit{handshake} (nawiązywanego przy zmianie protokołu \textit{HTTP} na \textit{Websocket} i
nawiązywaniu sesji z serwerem) które pozwalają przesłać token identyfikujący użytkownika i zezwolić bądź odmówić
zawiązanie połączenia w celu odbierania lub wysyłania wiadomości \cite{bib:stomp}.

\makesubsection{Przechowywanie plików z użyciem magazynu danych typu \textit{S3}}{sub:server-s3}

Omawiany system do przechowywania plików używa magazynu typu \textit{S3}. Magazyn ten składa się z \textit{kubełków}
(ang. \english{buckets}) oraz obiektów. \textit{Kubełki} reprezentują pojemniki na dane a obiekty składają się z klucza
(reprezentującego ścieżkę do pliku), danych oraz dodatkowych metadanych takich jak wielkość pliku, metody kodowania itp.
W projekcie skorzystano z dwóch serwerów \textit{S3} które przechowywały odpowiednio:
%
\begin{itemize}
  \item \textbf{pliki statyczne aplikacji} — tj. czcionki, grafiki używane w warstwie klienta oraz serwera i pliki
        tłumaczeń w formie dokumentów JSON,
  \item \textbf{pliki modyfikowane przez infrastrukturę serwera} — tj. zdjęcia profilowe użytkowników, \textit{Sfer},
        pliki dołączane do wiadomości tekstowych oraz tzw. \textit{lustrzane wiadomości email} (ang. \english{mirrored
          emails}) wykorzystywane w celu umożliwienia otwarcia ich w przeglądarce użytkownika.
\end{itemize}

Na etapie rozwoju projektu skorzystano z darmowego rozwiązania o nazwie \textit{Minio} uruchamianego w środowisku
\textit{Docker}. Udostępnia ono dwa interfejsy: jeden poprzez \textit{API} \textit{AWS SDK} a drugi jako aplikacja
webowa (widoczna na \imgref{fig:minio-interface}). Jako że \textit{Minio} posiada \textit{API} zgodne z rozwiązaniem
oferowanym przez \textit{AWS} — usługą \textit{S3} — w przypadku migracji produkcyjnej na chmurę \textit{AWS} nie będzie
potrzeby wprowadzania zmian w kodzie źródłowym aplikacji.
%
\begin{figure}[H]
  \centering
  \safeincludepng{3_3/6_1_minio_interface}{1}
  \caption{Interfejs webowy serwera typu \textit{S3} Minio.}
  \label{fig:minio-interface}
\end{figure}

\makesubsection{Zastosowane serwery baz danych oraz wzorzec \textit{DPS}}{sub:server-databases-dps}

W aplikacji skorzystano z trzech baz danych. Pierwsza z nich to relacyjna baza danych \textit{PostgreSQL}. Wykorzystano
ją jako główny system bazodanowy, w którym przechowywane jest większość informacji o użytkownikach, \textit{Sferach}
oraz wzajemnych relacji między nimi.

Drugą z kolei użytą bazą danych była nierelacyjna baza \textit{Redis}. Baza ta przechowuje dane w pamięci w postaci
klucz-wartość. Wykorzystana została do systemu cache umożliwiającego ograniczenie częstych zapytań do głównej bazy
danych \textit{PostgreSQL} na rzecz pobierania danych zapisanych w pamięci co jest istotne w kwestii budowania systemów
skierowanych na duże obciążenie \cite{bib:redis}.

Dodatkowo w celu przechowywania wiadomości wykorzystano nierelacyjną bazę danych \textit{Cassandra}. Jej główne cechy to
duża szybkość odczytu, skalowalność liniowa i duża odporność na błędy z uwagi na replikację danych. Wykorzystuje ona
język \textit{CQL}, podobny do języka \textit{SQL}, lecz pozbawiony jest on instrukcji \verb|JOIN|
\cite{bib:cassandra-overview}.

W przypadku klasycznej architektury monolitycznej występuje zazwyczaj jedna aplikacji i jedna baza danych. W takim
rozwiązaniu z oczywistych względów jest trudne (a wręcz czasami niemożliwie) skalowalnie
horyzontalne\footnote{Skalowanie horyzontalne — metoda zwiększania mocy obliczeniowej systemu poprzez dodawanie
  kolejnych instancji aplikacji działających równolegle \cite{bib:scaling-types}.} instancji bazy danych. W
przedstawianym projekcie skorzystano ze wzorca \textit{database-per-service} (skr. \textit{DPS}) (bez uwzględnienia bazy
\textit{Redis}), który narzuca, aby każdy mikroserwis korzystał z własnej instancji bazy danych. Podział baz danych
\textit{PostgreSQL} na tabele prezentuje się następująco:
%
\begin{itemize}
  \item baza danych \textbf{\texttt{user}} (\imgref{fig:pgsql-first-model}):
        \begin{itemize}
          \item tabela \textbf{\texttt{blacklist\_jwts}} — bufor przechowujący tokeny dostępu po wylogowaniu użytkownika
                (zakończeniu sesji),
          \item	tabela \textbf{\texttt{mfa\_users}} — dane do weryfikacji wieloetapowej wykonywanej poprzez aplikacje
                \textit{Microsoft Authenticator} lub \textit{Google Authenticator},
          \item tabela \textbf{\texttt{ota\_tokens}} — jednorazowe kody używane do resetowania hasła, aktywacji konta
                oraz weryfikacji wieloetapowej poprzez adres email,
          \item tabela \textbf{\texttt{refresh\_tokens}} — tokeny odświeżania przechowujące stan aktywnych sesji
                użytkownika w systemie,
          \item tabela \textbf{\texttt{roles}} — tabela ról użytkowników systemu,
          \item tabela \textbf{\texttt{users}} — podstawowe informacje o użytkowniku (imię, nazwisko, adres email, hasło
                itp.),
          \item tabela \textbf{\texttt{users\_roles}} — tabela łącząca użytkowników z rolami w relacji
                \texttt{WIELE-DO-WIELU},
        \end{itemize}
  \item baza danych \textbf{\texttt{sphere}} (\imgref{fig:pgsql-first-model}):
        \begin{itemize}
          \item tabela \textbf{\texttt{banned\_users}} — przechowywanie zablokowanych użytkowników \textit{Sfer},
          \item tabela \textbf{\texttt{guild\_links}} — przechowywanie linków z zaproszeniami do \textit{Sfer},
          \item tabela \textbf{\texttt{guilds}} — podstawowe informacje o \textit{Sferach} (nazwa, typ, identyfikator
                zarządcy \textit{Sfery} itp.),
          \item tabela \textbf{\texttt{text\_channels}} — kanały tekstowe \textit{Sfer},
          \item tabela \textbf{\texttt{user\_guilds}} — powiązanie użytkownika ze \textit{Sferą} (uczestnicy
                \textit{Sfery}),
        \end{itemize}
\end{itemize}
%
\begin{figure}[H]
  \centering
  \safeincludepng{3_3/7_1_pgsql_first_model}{1}
  \caption{Schematy tabel baz \textit{PostgreSQL} wykorzystanych w mikroserwisach \textit{user} i \textit{sphere}.}
  \label{fig:pgsql-first-model}
\end{figure}

\begin{itemize}
  \item baza danych \textbf{\texttt{multimedia}} (\imgref{fig:pgsql-second-model}):
        \begin{itemize}
          \item tabela \textbf{\texttt{account\_profiles}} — informacje o profilu użytkownika (kolor wiodący, ścieżka do
                niestandardowego zdjęcia itp.),
          \item tabela \textbf{\texttt{guild\_profiles}} — informacje o \textit{Sferze} (podobnie jak dla użytkownika,
                tj. kolor wiodący oraz ścieżka do niestandardowego zdjęcia),
        \end{itemize}
  \item baza danych \textbf{\texttt{notification}} (\imgref{fig:pgsql-second-model}):
        \begin{itemize}
          \item	tabela \textbf{\texttt{user\_notifs}} — informacje o ustawieniach powiadomień użytkownika,
        \end{itemize}
  \item baza danych \textbf{\texttt{oauth2\_client}} (\imgref{fig:pgsql-second-model}):
        \begin{itemize}
          \item tabela \textbf{\texttt{oauth2\_users}} — informacje o zarejestrowanym użytkowniku, którego konto do
                weryfikacji poświadczeń korzysta ze specyfikacji \textit{OpenID} (uwierzytelnianie społecznościowe z
                wykorzystaniem konta \textit{Google} lub \textit{Facebook}),
        \end{itemize}
  \item baza danych \textbf{\texttt{settings}} (\imgref{fig:pgsql-second-model}):
        \begin{itemize}
          \item tabela \textbf{\texttt{user\_relations}} — ustawienia powiązane z kontem użytkownika (wybrany język
                interfejsu, motyw kolorystyczny itp.).
        \end{itemize}
\end{itemize}
%
\begin{figure}[H]
  \centering
  \safeincludepng{3_3/7_2_pgsql_second_model}{1}
  \caption{Schematy pozostałych tabel w bazach danych \textit{PostgreSQL} mikroserwisów.}
  \label{fig:pgsql-second-model}
\end{figure}

Ponadto dla mikroserwisu \textit{chat} zastosowano pojedynczy klaster bazy danych \textit{Cassandra}. Klaster ten składa
się $N$ partycji, gdzie $N$ to liczba kanałów tekstowych. Wynika to z zastosowanego modelu danych widocznego na
\imgref{fig:chat-cassandra-model}. Znajdują się w nim 4 klucze proste tworzące klucz złożony. Klucze proste dzielą się
na klucze klastrowania i klucz partycjonowania. Klucz klastrowania odpowiada za sortowanie i wyszukiwanie rekordów w
obrębie partycji. Klucz partycjonowania natomiast rozdziela dane na osobne partycje \cite{bib:cassandra-data}. W
zrealizowanym systemie kluczem partycjonowania jest \verb|text_channel_id|, co powoduje stworzenie tylu partycji, ile
istnieje kanałów tekstowych. Pozostałe klucze proste to klucze klastrowania niezbędne do identyfikacji wiadomości lub
jej sortowania według daty (kolumna \verb|created_timestamp|) \cite{bib:cassandra-clustering}.
%
\begin{figure}[H]
  \centering
  \safeincludepng{3_3/7_3_chat_cassandra_model}{0.4}
  \caption{Model danych dla obsługi wiadomości czatu.}
  \label{fig:chat-cassandra-model}
\end{figure}

\makesubsection{
  Zachowanie integralności struktury bazy danych na poziomie aplikacji z wykorzystaniem systemu \textit{Liquibase}
}{sub:server-liquibase}

Poszczególne mikroserwisy w aplikacji korzystają z \textit{frameworka} \textit{Hibernate}. Jest to rozwiązanie
umożliwiające przekształcanie relacyjnych modelów danych na reprezentacje obiektowe w języku \textit{Java}. Domyślnie
framework ten sam zarządza modelem bazy danych, co może być mało elastyczne w przypadku modyfikowania tabel z
wprowadzonymi już danymi. Z tego względu w omawianym projekcie wyłączono całkowicie mechanizm automatycznego zarządzania
strukturą bazy danych na rzecz biblioteki \textit{Liquibase}. Rozwiązanie to przy użyciu kodu \textit{XML} i migracji
pozwala na ręczne zarządzanie modelem bazy danych, przy czym sam \textit{Hibernate} przy uruchomieniu aplikacji
weryfikuje jedynie poprawność modelu z jego zadeklarowaną reprezentacją obiektową. Dzięki zastosowaniu uniwersalnego
kodu \textit{XML} możliwe jest generowanie modelu niezależnie od dialektu języka \textit{SQL} dla wielu popularnych
rozwiązań \textit{RDBMS}. Na listingu \lisref{lis:liquibase} widoczny jest przykładowy skrypt usuwający dwie kolumny
\verb|password| i \verb|birth_date| oraz dodający je na nowo z uwzględnieniem nowych atrybutów (zezwolenie na wartości
typu \verb|nullable| w tych kolumnach).
%
\begin{lstlisting}[
  language=XML,
  label={lis:liquibase},
  caption={Przykładowy kod \textit{XML} migracji w systemie \textit{Liquibase}.}
]
<?xml version="1.0" encoding="UTF-8"?>
<databaseChangeLog
  xmlns="http://www.liquibase.org/xml/ns/dbchangelog"
  xmlns:xsi="http://www.w3.org/2001/XMLSchema-instance"
  xsi:schemaLocation="http://www.liquibase.org/xml/ns/dbchangelog
  https://www.liquibase.org/xml/ns/dbchangelog/dbchangelog-4.4.xsd">

  <property name="keyType" value="bigint unsigned" dbms="mysql,h2"/>
  <property name="keyType" value="bigint" dbms="postgresql,oracle"/>

  <changeSet id="2023-11-12_0000002_change-fields-to-nullable"
             author="milosz08" context="lq_dev, lq_docker">
    <dropColumn tableName="users" columnName="password"/>
    <dropColumn tableName="users" columnName="birth_date"/>

    <addColumn tableName="users">
      <column name="password" type="varchar(72)">
        <constraints nullable="true"/>
      </column>
      <column name="birth_date" type="date">
        <constraints nullable="true"/>
      </column>
    </addColumn>
  </changeSet>
</databaseChangeLog>
\end{lstlisting}

\makesubsection{Transakcje rozproszone z wykorzystaniem \textit{SAGA}}{sub:server-saga}

W realizowanym systemie z uwagi na działanie w środowisku rozproszonym zastosowano dwa rodzaje transakcji: lokalne i
rozproszone. Na poziomie samej pojedynczej metody mikroserwisu zachowanie reguł \textit{ACID} zapewniane jest przez
mechanizm \textit{Spring Transactional Management} i adnotację \verb|@Transactional|. Z kolei obsługa transakcji
rozproszony realizowana jest szynę danych \textit{Apache Kafka} i informację zwrotną w wysyłanym żądaniu. Jeśli dany
serwis będzie niedyspozycyjny lub zdarzy się inny nieoczekiwany błąd, uruchamiana jest transakcja kompensująca
wycofująca zmiany \cite{bib:saga}.

Na \imgref{fig:saga-pattern} widoczny jest przykład procesu biznesowego implementującego wzorzec \textit{SAGA}
przedstawiającego aktywację konta użytkownika w systemie podczas której to wystąpił błąd (błąd taki mógł wyniknąć z
chwilowej niedyspozycyjności usługi). W pierwszej kolejności sprawdzane jest, czy token aktywujący istnieje i jest
poprawny. Podczas wystąpienia błędu na etapie generacji ustawień następuje cofnięcie transakcji lokalnej oraz
kompensacja poprzedniej transakcji (w przedstawionym przykładzie kompensacja polega na przywróceniu stanu tokenu na
pierwotny, tj. nieużyty). Procedura kończy się komunikatem błędu zwracanym do użytkownika. Proces generowania zdjęcia w
tym wypadku nie nastąpi.
%
\begin{figure}[H]
  \centering
  \safeincludesvg{3_3/9_1_saga_pattern}{1}
  \caption{Zastosowanie \textit{SAGA} na przykładzie procesu aktywacji konta użytkownika.}
  \label{fig:saga-pattern}
\end{figure}

  \makechapter{Interfejs użytkownika}{chap:ui}

\makesection{Logowanie i rejestracja}{sec:ui-auth}

Na \imgref{fig:login-form} przedstawione jest okno logowania. Umożliwia ono zalogowanie standardowe oraz zalogowanie
przy użyciu konta \textit{Google} lub \textit{Facebook}. Użytkownik jako login może podać nazwę użytkownika lub adres
email przypisany do konta co sugeruje nazwa kontrolki formularza. Pola \textit{Nazwa użytkownika lub adres email} oraz
\textit{Hasło} zostały celowo rozdzielone, aby przeciwdziałać masowym próbom logowania dokonywanym przez boty w celu
unieruchomienia systemu.
%
\begin{figure}[H]
  \centering
  \safeincludepng{4_1/0_1_login_form}{0.9}
  \caption{Formularz logowania użytkownika.}
  \label{fig:login-form}
\end{figure}

Zalogowanie poprzez konto \textit{Google} lub \textit{Facebook} rozpoczyna się od kliknięcia widocznego przycisku na
\imgref{fig:login-form}. Przeglądarka przekierowuje użytkownika na stronę dostarczyciela poświadczeń. Po wprowadzeniu
danych i weryfikacji przez zewnętrzny serwer \textit{Google} lub \textit{Facebook} następuje przekierowanie powrotne do
aplikacji i (jeśli nie wystąpił błąd) zalogowanie użytkownika.

Sam formularz rejestracji (\imgref{fig:register-form}) podzielony został na dwa etapy, gdzie w pierwszym użytkownik
podaje najważniejsze dane identyfikujące go w systemie. W drugim etapie użytkownik może między innymi włączyć
uwierzytelnianie wieloetapowe które będzie musiał skonfigurować przy pierwszym logowaniu do systemu.
%
\begin{figure}[H]
  \centering
  \begin{subfigure}[b]{0.48\textwidth}
    \centering
    \safeincludepng{4_1/0_2_register_form_first}{1}
  \end{subfigure}
  \hfill
  \begin{subfigure}[b]{0.48\textwidth}
    \centering
    \safeincludepng{4_1/0_2_register_form_second}{1}
  \end{subfigure}
  \caption{Dwuetapowy formularz rejestracji.}
  \label{fig:register-form}
\end{figure}

Po wypełnieniu formularza rejestracji użytkownik musi poprawnie przejść weryfikację Captcha. Zaimplementowano ją w celu
przeciwdziałania masowego tworzeniu pustych kont przez boty (\imgref{fig:register-captcha}). Gdy rejestracja przebiegła
pomyślnie, na podaną skrzynkę email wysyłany jest kod aktywujący konto (\imgref{fig:activate-email-form}). Konto takie
można aktywować w przeciągu 48 godzin. Nieaktywowane konto po tym czasie zostanie usunięte z systemu.
%
\begin{figure}[H]
  \centering
  \safeincludepng{4_1/0_3_register_captcha}{0.7}
  \caption{Okno z polem \textit{Captcha}.}
  \label{fig:register-captcha}
\end{figure}
%
\begin{figure}[H]
  \centering
  \begin{subfigure}[b]{0.53\textwidth}
    \centering
    \safeincludepng{4_1/0_4_activate_email_first}{1}
  \end{subfigure}
  \hfill
  \begin{subfigure}[b]{0.43\textwidth}
    \centering
    \safeincludepng{4_1/0_4_activate_form_second}{1}
  \end{subfigure}
  \caption{Email z potwierdzeniem aktywacji konta oraz formularz aktywacji.}
  \label{fig:activate-email-form}
\end{figure}

\makesection{Uwierzytelnianie wieloetapowe}{sec:ui-auth-mfa}

Gdy użytkownik podczas rejestracji zaznaczył opcję włączenia uwierzytelniania wieloetapowego, przy pierwszym logowaniu
zostanie poproszony o dokończenie konfiguracji. Odbywa się to poprzez skan kodu QR lub ręcznym zarejestrowaniem tajnego
klucza w aplikacji \textit{Authenticator}. Klucz taki można dodatkowo pobrać w formie pliku tekstowego przy użyciu
przycisku z etykietą \textit{Zapisz do pliku}. Pierwsze etapy konfigurowania uwierzytelniania wieloskładnikowego
widoczne są na \imgref{fig:mfa-settings}.
%
\begin{figure}[H]
  \centering
  \begin{subfigure}[b]{0.48\textwidth}
    \centering
    \safeincludepng{4_2/0_1_mfa_settings_first}{1}
  \end{subfigure}
  \hfill
  \begin{subfigure}[b]{0.48\textwidth}
    \centering
    \safeincludepng{4_2/0_1_mfa_settings_second}{1}
  \end{subfigure}
  \caption{Pierwszy etap konfiguracji uwierzytelniania wieloskładnikowego.}
  \label{fig:mfa-settings}
\end{figure}

Po udanej konfiguracji, aplikacja \textit{Microsoft Authenticator} sekwencyjnie co pewien okres będzie generowała kody.
Przy kolejnym logowaniu będzie należało podać ważny kod z aplikacji. Na \imgref{fig:mfa-settings-third} przedstawiono
okno konfiguracji a na \imgref{fig:mfa-authenticator-app} zrzut ekranu telefonu z aplikacją \textit{Microsoft
  Authenticator}. Aplikacja \textit{Visphere} jest zarejestrowana i co 30 sekund generowany jest nowy kod możliwy do
użycia w celu dodatkowego potwierdzenia tożsamości przy procedurze logowania. W przypadku utracenia dostępu do aplikacji
\textit{Authenticator} możliwe jest również zalogowanie się z wykorzystaniem jednorazowego tokenu wysyłanego w
wiadomości email i ponowne jej skonfigurowanie.
%
\begin{figure}[H]
  \centering
  \safeincludepng{4_2/0_2_mfa_settings_third}{0.42}
  \caption{Końcowy etap konfiguracji uwierzytelniania wieloetapowego.}
  \label{fig:mfa-settings-third}
\end{figure}
%
\begin{figure}[H]
  \centering
  \safeincludepng{4_2/0_3_mfa_authenticator_app}{0.42}
  \caption{Okno aplikacji \textit{Microsoft Authenticator}.}
  \label{fig:mfa-authenticator-app}
\end{figure}

\makesection{Reset hasła z wykorzystaniem jednorazowego kodu}{sec:ui-password-reset}

W celu zresetowania hasła należy w formularzu (\imgref{fig:password-reset}) wpisać nazwę użytkownika lub adres email,
aby serwer wysłał wiadomość z kodem (\imgref{fig:password-reset-change}).
%
\begin{figure}[H]
  \centering
  \begin{subfigure}[b]{0.45\textwidth}
    \centering
    \safeincludepng{4_3/0_1_password_reset_first}{1}
  \end{subfigure}
  \hfill
  \begin{subfigure}[b]{0.45\textwidth}
    \centering
    \safeincludepng{4_3/0_1_password_reset_second}{1}
  \end{subfigure}
  \caption{Pierwszy etap konfiguracji uwierzytelniania wieloskładnikowego.}
  \label{fig:password-reset}
\end{figure}
%
\begin{figure}[H]
  \centering
  \begin{subfigure}[b]{0.45\textwidth}
    \centering
    \safeincludepng{4_3/0_2_password_change_email}{1}
  \end{subfigure}
  \hfill
  \begin{subfigure}[b]{0.45\textwidth}
    \centering
    \safeincludepng{4_3/0_2_password_change}{1}
  \end{subfigure}
  \caption{Pierwszy etap konfiguracji uwierzytelniania wieloskładnikowego.}
  \label{fig:password-reset-change}
\end{figure}

Po wysłaniu prośby o zmianę hasła aplikacja nie informuje użytkownika, czy dana osoba istnieje a jedynie (ze względów
bezpieczeństwa) podaje informację, że podjęto próbę wysłania wiadomości. Kod z wiadomości email należy przepisać w
formularzu (\imgref{fig:password-reset}) lub kliknąć w link. System wówczas przekieruje na stronę z formularzem
umożliwiającym zmianę hasła (\imgref{fig:password-reset-change}). Sama procedura zmiany hasła dostępna jest jedynie dla
kont lokalnym tj. założonych w aplikacji przy użyciu standardowego formularza rejestracji widocznego na
\imgref{fig:register-form}.

\makesection{Edycja profilu użytkownika}{sec:ui-profile-edit}

Na \imgref{fig:user-settings} widoczny jest fragment ustawień użytkownika umożliwiający zmianę podstawowych parametrów
konta, tj. zmiana hasła, włączenie lub wyłączenie weryfikacji dwuetapowej oraz wyłączenie i usunięcie konta. Konto
wyłączone w każdej chwili można przywrócić, a dane oraz ustawienia użytkownika nie są tracone. Przycisk usunięcia konta
usuwa je trwale z systemu. Dodatkowo użytkownik może zdecydować, czy po usunięciu lub zablokowaniu konta wszystkie jego
wiadomości ze wszystkich kanałów tekstowych mają zostać usunięte. W przypadku nieusunięcia wiadomości, ich autor będzie
oznaczony jako \textit{Użytkownik usunięty}.
%
\begin{figure}[H]
  \centering
  \safeincludepng{4_4/0_1_user_settings}{0.8}
  \caption{Fragment ustawień konta użytkownika.}
  \label{fig:user-settings}
\end{figure}

Z kolei na \imgref{fig:user-settings-colors} widoczne są ustawienia wyglądu profilu i możliwość wgrania własnego
zdjęcia, wygenerowania identikonu\footnote{Identikon, z ang. \english{identicon} — geometryczny obrazek generowany na
  podstawie wartości zdefiniowanych przez funkcję skrótu wykonywaną na nazwie użytkownika aplikacji.} oraz usunięcia
zdjęcia (przywrócenie domyślnego zdjęcia z inicjałami). Wgrywanie zdjęcia odbywa się poprzez okno umożliwiające
załadowanie grafiki i wykadrowanie jej przed wysłaniem na serwer (\imgref{fig:add-crop-image}).
%
\begin{figure}[H]
  \centering
  \begin{subfigure}[b]{0.45\textwidth}
    \centering
    \safeincludepng{4_4/0_3_add_image}{1}
  \end{subfigure}
  \hfill
  \begin{subfigure}[b]{0.45\textwidth}
    \centering
    \safeincludepng{4_4/0_3_crop_image}{1}
  \end{subfigure}
  \caption{Okno umożliwiające wgranie i wykadrowanie zdjęcia profilowego.}
  \label{fig:add-crop-image}
\end{figure}
%
\begin{figure}[H]
  \centering
  \safeincludepng{4_4/0_2_user_settings_colors}{0.9}
  \caption{Ustawienia profilu użytkownika (zdjęcie oraz kolor wiodący).}
  \label{fig:user-settings-colors}
\end{figure}

\makesection{Wsparcie dla wielu języków (internacjonalizacja aplikacji)}{sec:ui-i18n}

System posiada pełne wsparcie z zakresu internacjonalizacji dla języka polskiego oraz angielskiego w dialekcie
amerykańskim. Oprócz wyświetlania samego interfejsu użytkownika w wybranym języku dostosowuje on również format dat do
wybranego kraju. Widoczne to jest na \imgref{fig:lang-comparison}. Każdy użytkownik może przypisać wybrany język do
konta w aplikacji, dzięki czemu, gdy zaloguje się na innym urządzeniu na to konto, interfejs będzie wyświetlany zgodnie
z wybranym wcześniej językiem (\imgref{fig:change-language}).
%
\begin{figure}[H]
  \centering
  \begin{subfigure}[b]{0.40\textwidth}
    \centering
    \safeincludepng{4_5/0_1_lang_comparison_en}{1}
  \end{subfigure}
  \hfill
  \begin{subfigure}[b]{0.40\textwidth}
    \centering
    \safeincludepng{4_5/0_1_lang_comparison_pl}{1}
  \end{subfigure}
  \caption{Wyświetlanie treści oraz dat w różnych językach.}
  \label{fig:lang-comparison}
\end{figure}
%
\begin{figure}[H]
  \centering
  \safeincludepng{4_5/0_2_change_language}{0.9}
  \caption{Zakładka umożliwiająca zmianę ustawień językowych.}
  \label{fig:change-language}
\end{figure}

\makesection{Główne okno aplikacji}{sec:ui-main-window}

Główne okno aplikacji podzielone jest na kilka modułów (\imgref{fig:main-view}). Moduły 1 oraz 4 są niezmienne dla
wszystkich \textit{Sfer} natomiast zawartość modułów od 2, 3 oraz 5 zmienia się w zależności od wybranej \textit{Sfery}.
Poszczególne moduły reprezentują:
%
\begin{enumerate}
  \item moduł wszystkich \textit{Sfer} użytkownika (własne oraz te do których dołączył),
  \item moduł kanałów tekstowych wybranej \textit{Sfery},
  \item moduł wiadomości wybranego kanału tekstowego z modułu 2,
  \item moduł pola wprowadzania wiadomości oraz opcjonalnych plików (załączników),
  \item moduł widocznych członków \textit{Sfery} z wyodrębnieniem właściciela.
\end{enumerate}

Dodatkowo na \imgref{fig:main-view} widoczne jest okno typu popup. Otwiera się je po kliknięciu na wybranego użytkownika
z modułu 5. Jeśli wybrany użytkownik jest członkiem \textit{Sfery}, to dla administratora widoczne są przyciski które
umożliwiają jego wyrzucenie, zablokowanie oraz oddelegowanie \textit{Sfery}. Procedura oddelegowania polega na
przekazaniu praw innemu użytkownikowi, który wówczas staje się administratorem serwera. Funkcja ta może zostać użyta w
przypadku chęci usunięcia konta w aplikacji przez właściciela \textit{Sfer}, ale bez usuwania tych \textit{Sfer}
(program nie pozwoli usunąć konta, jeśli użytkownik będzie zarządcą przynajmniej jednej \textit{Sfery}).
%
\begin{figure}[H]
  \centering
  \safeincludepng{4_6/0_1_main_view}{1}
  \caption{Główne okno aplikacji z wydzielonymi modułami.}
  \label{fig:main-view}
\end{figure}

\makesection{Wysyłanie wiadomości tekstowych wraz z załączonymi plikami}{sec:ui-messages-with-files}

Aplikacja zgodnie z założeniami umożliwia dołączanie do wiadomości plików. W jednej wiadomości może być maksymalnie 5
plików z czego każdy plik może ważyć maksymalne 20 MB. Jak widać na \imgref{fig:append-file} pliki graficzne wyświetlane
są jako podgląd. Pozostałe pliki widoczne są jedynie w formie bloku z typem pliku oraz przyciskiem umożliwiającym
pobranie ich na komputer.
%
\begin{figure}[H]
  \centering
  \safeincludepng{4_7/0_1_append_file}{1}
  \caption{Pole wprowadzania wiadomości tekstowej wraz z widocznymi plikami.}
  \label{fig:append-file}
\end{figure}

Dodawanie plików odbywa się poprzez przycisk \textit{+} lub wykonanie procedury wklejania zasobu przy użyciu skrótu
klawiaturowego \verb|Ctrl+V| w polu wpisywania wiadomości. Wszystkie zasoby dodawane do wiadomości pojawiają się w
dolnej przewijanej liście i możliwa jest ich edycja (usunięcie lub dodanie nowego zasobu) przed opublikowaniem
wiadomości (\imgref{fig:main-view-files}). Każdą wysłaną wiadomość może usunąć jej autor oraz administrator
\textit{Sfery}.
%
\begin{figure}[H]
  \centering
  \safeincludepng{4_7/0_2_main_view_files}{1}
  \caption{Główne okno kanału tekstowego z widocznymi wiadomościami w formie plików.}
  \label{fig:main-view-files}
\end{figure}

\makesection{Tworzenie i dołączanie do \textit{Sfery}}{sec:ui-create-and-join-sphere}

Po zalogowaniu użytkownik ma możliwość stworzenia \textit{Sfery} (serwera zawierającego kanały komunikacyjne). Może
określić jej nazwę, kategorię oraz widoczność (\imgref{fig:create-sphere}). Do \textit{Sfery} prywatnej mogą dołączać
tylko osoby z zaproszeniem. Do \textit{Sfery} publicznej może dołączyć każdy posiadający konto w aplikacji. Na
\imgref{fig:join-links} widoczna jest zakładka do generowania linków umożliwiających dołączenie. Każdy taki link może
mieć określony czas wygaśnięcia oraz nazwę. Dodatkowo link można tymczasowo wyłączyć. Kod z linku może zostać
wykorzystany w oknie umożliwiającym dołączenie do Sfery (\imgref{fig:join-to-sphere}).
%
\begin{figure}[H]
  \centering
  \safeincludepng{4_8/0_1_create_sphere}{0.6}
  \caption{Okno umożliwiające stworzenie \textit{Sfery}.}
  \label{fig:create-sphere}
\end{figure}
%
\begin{figure}[H]
  \centering
  \safeincludepng{4_8/0_2_join_to_sphere}{0.5}
  \caption{Okno umożliwiające dołączenie do prywatnej \textit{Sfery} z użyciem kodu.}
  \label{fig:join-to-sphere}
\end{figure}
%
\begin{figure}[H]
  \centering
  \safeincludepng{4_8/0_3_join_links}{0.9}
  \caption{Zakładka wygenerowanych linków.}
  \label{fig:join-links}
\end{figure}

\makesection{Tworzenie kanału tekstowego}{sec:ui-create-text-channel}

W każdej \textit{Sferze} zarządca możne tworzyć, modyfikować oraz usuwać kanały tekstowe. W celu stworzenia kanału
tekstowego należy podać jego nazwę (\imgref{fig:create-text-channel}). Każdy taki kanał można usunąć wraz ze wszystkimi
wiadomościami znajdującymi się na nim. Po stworzeniu, kanał tekstowy pojawi się w module 2 (\imgref{fig:main-view}).
%
\begin{figure}[H]
  \centering
  \safeincludepng{4_9/0_1_create_text_channel_modal}{0.6}
  \caption{Okno umożliwiające stworzenie \textit{Sfery}.}
  \label{fig:create-text-channel}
\end{figure}

  \makechapter{Weryfikacja i walidacja}{chap:verification-and-validation}

\makesection{Testy aplikacji klienta}{sec:client-tests}

Testy klienta w głównej mierze polegały na manualnym sprawdzeniu poprawności działania zaimplementowanych
funkcjonalności uwzględniając poprawne wyświetlanie interfejsu użytkownika oraz walidację formularzy. W bardziej
znaczących miejscach zdecydowano się napisać testy jednostkowe z wykorzystaniem biblioteki \textit{Jasmine} i
\textit{Karma}. Przykładowy test sprawdzający poprawność sanityzacji\footnote{Sanityzacja (ang. \english{sanitization})
  — usunięcie z danych wejściowych (najczęściej wprowadzanych przez użytkownika) elementów, które mogą wprowadzić
  niezamierzone zmiany w systemie \cite{bib:validation-and-sanitation}.} wartości wejściowych widoczny jest na listingu
\lisref{lis:client-test}. Trzy najważniejsze funkcje to \verb|describe|, \verb|beforeEach| oraz \verb|it|. Funkcja
\verb|describe| odpowiada za identyfikację testu pośród innych testów jednostkowych aplikacji. Funkcja \verb|it|
symbolizuje pojedynczą jednostkę testującą, w tym przypadku usunięcie potencjalnie szkodliwego kodu z ciągu znaków.
Funkcja \verb|beforeEach| natomiast wykonuje się przed wszystkimi jednostkami testującymi, umożliwiając ustawienie
początkowych parametrów testowanego komponentu. W przykładowym teście w funkcji \verb|beforeEach| tworzony jest moduł
testowy zawierający główny moduł aplikacji \verb|AppModule| oraz tworzona jest nowa instancja testowanego komponentu.
Natomiast w funkcji \verb|it| przy użyciu procedury \verb|expect| przyrównywane są dwie wartości do siebie w celu
weryfikacji poprawności przeprowadzonej sanityzacji.
%
\begin{lstlisting}[style=JSES6Base,label={lis:client-test},caption={Przykładowy test jednostkowy w aplikacji klienta}]
describe('SanitationPipe', () => {
  let pipe: SanitationPipe;
  beforeEach(() => {
    TestBed.configureTestingModule({
      imports: [AppModule],
    }).compileComponents();
    pipe = new SanitationPipe(TestBed.inject(DomSanitizer));
  });
  it('should remove scripts value', () => {
    const mockedExploit = "
      <script>
        alert('hello! I am very angry exploit :)')
      </script>
      <h2>I'm polite.</h2>
    ";
    const result = pipe.transform(mockedExploit);
    expect(result).toBe("<h2>I'm polite.</h2>");
  });
});
\end{lstlisting}

Na \imgref{fig:karma-output} widoczny jest wynik wszystkich testów aplikacji klienta. Każdy komponent posiada
przynajmniej jeden test podstawowy sprawdzający poprawność jego tworzenia, stąd tak duża ich ilość. Co więcej, dzięki
wykorzystaniu narzędzia \textit{Husky}, system nie pozwoli umieścić zmian na zdalnym repozytorium \textit{GIT}, jeśli
przynajmniej jeden test nie zostanie pozytywnie zakończony. Eliminuje to możliwość wprowadzenia wadliwie działającego
kodu na etapie rozwoju aplikacji przed umieszczeniem jej na serwerze produkcyjnym.
%
\begin{figure}[H]
  \centering
  \safeincludepng{5_1/0_1_karma_output}{1}
  \caption{Wynik wszystkich testów aplikacji klienta w konsoli.}
  \label{fig:karma-output}
\end{figure}

\makesection{Testy architektury serwera}{sec:server-tests}

Większość architektury serwera (punkty końcowe \textit{REST}) zostały przetestowane z użyciem narzędzia
\textit{Postman}. Jest to program umożliwiający sprawdzenie poprawności zapytań \textit{HTTP}. Jego specyfika działania
powoduje, że jest on również interaktywną dokumentacją do projektu. Umożliwia deklarowanie zmiennych, które mogą
przyjmować inne wartości dla różnych środowisk oraz wprowadza możliwość uwierzytelniania zasobów przy pomocy
\textit{JWT} co dodatkowo usprawnia testowanie. Same zapytania przechowywane są w folderach, wprowadzając porządek, co
jest istotne przy dużych projektach programistycznych. Na \imgref{fig:postman} widoczne jest okno programu
\textit{Postman} z przykładowym zapytaniem \textit{HTTP} i jego rezultatem.
%
\begin{figure}[H]
  \centering
  \safeincludepng{5_2/0_1_postman}{1}
  \caption{Okno programu \textit{Postman} z widocznym zapytaniem \textit{HTTP}.}
  \label{fig:postman}
\end{figure}

Dodatkowo do poszczególnych fragmentów kodu w mikroserwisach zostały napisane testy jednostkowe z wykorzystaniem
bibliotek \textit{Spring Boot Starter Test} oraz \textit{JUnit}. Na listingu \lisref{lis:java-test} widoczny jest
przykład testu jednostkowego sprawdzającego poprawność działania metody konwertującej nazwę \textit{Sfery} lub
użytkownika na inicjały wyświetlane w formie grafiki (domyślne zdjęcie profilowe).
%
\begin{lstlisting}[
  language=Java,
  label={lis:java-test},
  caption={Test jednostkowy metody \texttt{parseGuildNameInitials()}.}
]
@Test
void parseGuildNameInitials() {
  // given
  final String[] mockedNames = {
    "testguild123", "test test123", "GGD2"
  };
  final String[] expectedResults = { "t", "tt", "G" };

  // when
  final String[] actualResults = new String[expectedResults.length];
  for (int i = 0; i < mockedNames.length; i++) {
    actualResults[i] = new String(
      StringParser.parseGuildNameInitials(mockedNames[i])
    );
  }

  // then
  assertArrayEquals(expectedResults, actualResults);
}
\end{lstlisting}

Metoda ta (listing \lisref{lis:java-method}) rozdziela ciąg wejściowy względem delimitera (w tym przypadku znaku spacji)
a następnie pobiera pierwszy znak każdego rozdzielonego elementu. Jeśli ilość elementów jest mniejsza od dwóch, pobiera
jedynie pierwszy znak. Każdy test jednostkowy w aplikacji podzielony jest na 3 etapy. Są to odpowiednio \verb|given|,
\verb|when| i \verb|then|. \verb|Given| symbolizuje dane wejściowe oraz oczekiwany rezultat, \verb|when| jest
uruchomieniem funkcji testującej a \verb|then| sprawdzeniem wyników. Jeśli dane wyjściowe nie będą zgodnie z
oczekiwanymi, test zwróci błąd z widocznym stosem wywołań (\imgref{fig:junit-error}). Jeśli natomiast test zostanie
zakończony pomyślnie, zostanie zwrócona wartość 0 oznaczająca zakończenie operacji sukcesem.
%
\begin{lstlisting}[language=Java,label={lis:java-method},caption={Ciało metody \texttt{parseGuildNameInitials()}.}]
public static char[] parseGuildNameInitials(String guildName) {
  final String[] parts = guildName.split(StringUtils.SPACE);
  final char[] initials;
  if (parts.length > 1) {
    initials = new char[2];
    for (int i = 0; i < 2; i++) {
      initials[i] = parts[i].charAt(0);
    }
  } else {
    initials = new char[]{ parts[0].charAt(0) };
  }
  return initials;
}
\end{lstlisting}
%
\begin{figure}[H]
  \centering
  \safeincludepng{5_2/0_2_junit_error}{1}
  \caption{Wynik testu z widocznym błędem.}
  \label{fig:junit-error}
\end{figure}

\makesection{Napotkane problemy}{sec:issues}

\makesubsection{
  Przekazywanie stanu pomiędzy sesje Websocket w wielu instancjach mikroserwisu \textit{chat-service}
}{sub:ws-state-issue}

W protokole \textit{Websocket} w przeciwieństwie do \textit{HTTP} nie jest możliwym zaprojektowania komunikacji
bezstanowej z uwagi na sposób jego działania (ciągłość działania poprzez otwarty kanał komunikacyjny z serwerem). Z tego
powodu przy uruchomionych wielu instancjach mikroserwisu czatu, klient mógł się przyłączyć do sesji \textit{Websocket}
do której nie był przyłączony odbiorca, co skutkowało brakiem propagowania wiadomości kanału tekstowego. W celu
rozwiązania tego problemu skorzystano z usługi propagacji przy użyciu brokera wiadomości \textit{RabbitMQ}. W działaniu
przypomina oprogramowanie \textit{Apache Kafka}, lecz w przeciwieństwie do używania własnego protokoły komunikacji
opartego o \textit{TCP}, implementuje on interfejs \textit{JMS} oraz wspiera protokół \textit{STOMP}, co jest kluczową
zaletą z uwagi na wykorzystanie jego przy komunikacji z użyciem protokołu \textit{Websocket}. Sam \textit{RabbitMQ}
pośredniczy pomiędzy klientem \textit{Websocket} a serwerem tworząc pule połączeń oraz propagując wiadomości wysyłane
przez użytkownika do wszystkich aktywnych sesji w wielu instancjach mikroserwisu, a co za tym idzie aktywnych
użytkowników zainteresowanych odbiorem (na tym samym kanale tekstowym) \cite{bib:real-time-app-scaling}.

  \makechapter{Podsumowanie i wnioski}{chap:outro}

W zrealizowanym projekcie wykorzystano szereg technologii umożliwiających stworzenie wydajnego i przygotowanego na
skalowanie horyzontalne systemu czatu. Zastosowana architektura mikroserwisów pozwoliła na rozdzielenie funkcjonalne i
izolację składowych fragmentów systemu co znacznie ułatwi późniejszą jego rozbudowę. Co więcej architektura taka jest
powszechnie używana w średnich i dużych systemach komercyjnych z uwagi na szereg zalet jakie oferuje. Jedną z nich jest
możliwość uruchamiania wielu instancji tego samego mikroserwisu w celu rozłożenia obciążenia generowanego przez wielu
użytkowników jednocześnie. Dodatkowo każdy taki serwis posiada osobną bazę danych, co uniezależnia pojedyncze
podjednostki systemu i umożliwia dodatkową jej replikację, jeśli jedna okaże się niewystarczająca. Zastosowana baza
danych \textit{Cassandra} w systemie czatu jest znana ze swojej odporności na błędy dzięki powielaniu danych w kilku
węzłach skutecznie eliminując możliwość ich utraty \cite{bib:cassandra-overview}. Dodatkowo zastosowany w nich mechanizm
partycjonowania pozwala na równomierny rozkład danych na zdecentralizowanych serwerach. W celu zapewnienia szybkiej i
niezawodnej komunikacji między elementami infrastruktury serwera skorzystano z metody sterowanej zdarzeniami
wykorzystującej oprogramowanie \textit{Apache Kafka} co znacznie przyspieszyło i zwiększyło możliwości skalowania
takiego systemu. Z kolei w komunikacji z klientem skorzystano z protokołów \textit{Websocket} oraz \textit{STOMP}
umożliwiających dwukierunkową wymianę wiadomości w czasie rzeczywistym, co jest kluczowym aspektem w nowoczesnych i
interaktywnych aplikacjach czatowych.

Oprócz szeregu zalet, największą wadą architektury mikroserwisowej zaobserwowaną podczas tworzenia systemu jest jej
stosunkowo duża złożoność. Wynika to z potrzeby stosowania mechanizmów które nie występują (lub występują w nieznacznym
stopniu) w typowej architekturze monolitycznej. Skutkiem posiadania niezbędnych dodatkowych usług jest wysokie
zapotrzebowanie na moc obliczeniową co może przyczynić się do zwiększenia kosztów utrzymania infrastruktury. Samo
zastosowanie \textit{Apache Kafka} i architektury sterowanej zdarzeniami zamiast standardowych zapytań \textit{HTTP} w
komunikacji między komponentami serwera generuje dodatkowe koszty obliczeniowe z uwagi na potrzebę uruchomionego brokera
\textit{Kafka} i serwera \textit{Zookeeper} jedocześnie oraz może stanowić wyzwanie dla nowych programistów
wkraczających w projekt z uwagi na większy stopień skomplikowania \cite{bib:kafka-overview}. Mimo wszystko odpowiednio
wdrożona architektura mikroserwisowa w środowisku produkcyjnym może obsłużyć setki tysięcy użytkowników jednocześnie.

\makesection{Perspektywy rozwoju}{sec:futher-progress}

W celu rozbudowy systemu i poprawy komfortu użytkowników przewiduje się stworzenie aplikacji mobilnej oraz desktopowej
posiadającej większość funkcjonalności zawartych w konwencjonalnej wersji dostępnej na przeglądarkę internetową.
Dodatkowo rozważane jest stworzenie dodatkowych elementów w interfejsie użytkownika takich jak: informacja o aktualnym
statusie (online, offline, nie przeszkadzać itp.) oraz system rang umożliwiający dodawanie do nich wybranych uprawnień i
przydzielanie tych rang do użytkowników.

Architektura systemu w stworzonej konfiguracji korzysta z konteneryzacji \textit{Docker}. Dobrym kolejnym krokiem byłoby
skonfigurowanie do niej środowiska orkiestracji kontenerów takiego jak np. \textit{Kubernetes} w celu automatycznego
uruchamiania replik instancji mikroserwisów w przypadku zwiększenia ruchu. Dodatkowo przewiduje się wprowadzenie jednego
spójnego zcentralizowanego systemu do zapisu logów aplikacji korzystającego ze stosu \textit{ELK}
(\textit{ElasticSearch}, \textit{LogStash}, \textit{Kibana}). Rozważa się również przyszłe umieszczenie systemu w
usłudze \textit{Cloud Computing} takiej jak \textit{AWS} oraz zaimplementowanie procedury \textit{CI/CD} przy pomocy
rozwiązania \textit{Terraform} oferowanego przez firmę \textit{HashiCorp}.

}{
  % dodatki
  \makechapter{Spis skrótów i symboli}{chap:shortcuts-and-symbols}

\begin{longtable}[l]{ l @{~~--~~} p{376pt} }
  MFA   & uwierzytelnianie wieloskładnikowe (wieloetapowe) (ang. \english{Multi Factor Authentication})           \\
  JVM   & wirtualna maszyna Javy (ang. \english{Java Virtual Machine})                                            \\
  SPA   & jednostronicowa aplikacja internetowa (ang. \english{Single Page Application})                          \\
  IDE   & zintegrowane środowisko programistyczne (ang. \english{Integrated Development Environment})             \\
  REST  & bezstanowa architektura komunikacji dla protokołu HTTP (ang. \english{Representation State Transfer})   \\
  XML   & rozszerzony język znaczników (ang. \english{eXtensible Markup Language})                                \\
  RDBMS & system zarządzania relacyjną bazą danych (ang. \english{Relational Database Management System})         \\
  SQL   & język zapytań relacyjnej bazy danych (ang. \english{Structured Query Language})                         \\
  CQL   & język zapytań w systemie bazy danych Cassandra (ang. \english{Cassandra Query Language})                \\
  ACID  & atomowość, spójność, izolacja, trwałość (ang. \english{Atomicity, Consistency, Isolation, Durability})  \\
  API   & interfejs programistyczny aplikacji (ang. \english{Application Programming Interface})                  \\
  S3    & serwer przechowujący pliki statyczne (obrazy, pliki audio, itp.) (ang. \english{Simple Storage System}) \\
  CDN   & rozproszona sieć serwerów dostarczających treść (ang. \english{Content Delivery Network})               \\
  AJAX  & asynchroniczny JavaScript i XML (ang. \english{Asynchronus Javascript And XML})                         \\
  JSON  & sposób zapisu obiektów w formie ciągu znaków (ang. \english{Javascript Object Notation})                \\
  JWT   & token dostępu wykorzystujący specyfikację JSON (ang. \english{JSON Web Token})                          \\
  CI/CD & ciągła integracja/ciągłe dostarczanie (ang. \english{Continuous Integration/Continuous Delivery})       \\
  AWS   & usługa udostępniania mocy obliczeniowej w chmurze (ang. \english{Amazon Web Services})                  \\
  JMS   & interfejs przesyłania i odbioru komunikatów w środowisku JVM (ang. \english{Java Message Service})      \\
  SDK   & zestaw narzędzi programistycznych (ang. \english{Software Development Kit})                             \\
\end{longtable}

}
