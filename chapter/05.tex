\makechapter{Weryfikacja i walidacja}{chap:verification-and-validation}

\makesection{Testy aplikacji klienta}{sec:client-tests}

Testy klienta w głównej mierze polegały na manualnym sprawdzeniu poprawności działania zaimplementowanych
funkcjonalności uwzględniając poprawne wyświetlanie interfejsu użytkownika oraz walidację formularzy. W bardziej
znaczących miejscach zdecydowano się napisać testy jednostkowe z wykorzystaniem biblioteki \textit{Jasmine} i
\textit{Karma}. Przykładowy test sprawdzający poprawność sanityzacji\footnote{Sanityzacja (ang. \english{sanitization})
  — usunięcie z danych wejściowych (najczęściej wprowadzanych przez użytkownika) elementów, które mogą wprowadzić
  niezamierzone zmiany w systemie \cite{bib:validation-and-sanitation}.} wartości wejściowych widoczny jest na listingu
\lisref{lis:client-test}. Trzy najważniejsze funkcje to \verb|describe|, \verb|beforeEach| oraz \verb|it|. Funkcja
\verb|describe| odpowiada za identyfikację testu pośród innych testów jednostkowych aplikacji. Funkcja \verb|it|
symbolizuje pojedynczą jednostkę testującą, w tym przypadku usunięcie potencjalnie szkodliwego kodu z ciągu znaków.
Funkcja \verb|beforeEach| natomiast wykonuje się przed wszystkimi jednostkami testującymi, umożliwiając ustawienie
początkowych parametrów testowanego komponentu. W przykładowym teście w funkcji \verb|beforeEach| tworzony jest moduł
testowy zawierający główny moduł aplikacji \verb|AppModule| oraz tworzona jest nowa instancja testowanego komponentu.
Natomiast w funkcji \verb|it| przy użyciu procedury \verb|expect| przyrównywane są dwie wartości do siebie w celu
weryfikacji poprawności przeprowadzonej sanityzacji.
%
\begin{lstlisting}[style=JSES6Base,label={lis:client-test},caption={Przykładowy test jednostkowy w aplikacji klienta}]
describe('SanitationPipe', () => {
  let pipe: SanitationPipe;
  beforeEach(() => {
    TestBed.configureTestingModule({
      imports: [AppModule],
    }).compileComponents();
    pipe = new SanitationPipe(TestBed.inject(DomSanitizer));
  });
  it('should remove scripts value', () => {
    const mockedExploit = "
      <script>
        alert('hello! I am very angry exploit :)')
      </script>
      <h2>I'm polite.</h2>
    ";
    const result = pipe.transform(mockedExploit);
    expect(result).toBe("<h2>I'm polite.</h2>");
  });
});
\end{lstlisting}

Na \imgref{fig:karma-output} widoczny jest wynik wszystkich testów aplikacji klienta. Każdy komponent posiada
przynajmniej jeden test podstawowy sprawdzający poprawność jego tworzenia, stąd tak duża ich ilość. Co więcej, dzięki
wykorzystaniu narzędzia \textit{Husky}, system nie pozwoli umieścić zmian na zdalnym repozytorium \textit{GIT}, jeśli
przynajmniej jeden test nie zostanie pozytywnie zakończony. Eliminuje to możliwość wprowadzenia wadliwie działającego
kodu na etapie rozwoju aplikacji przed umieszczeniem jej na serwerze produkcyjnym.
%
\begin{figure}[H]
  \centering
  \safeincludepng{5_1/0_1_karma_output}{1}
  \caption{Wynik wszystkich testów aplikacji klienta w konsoli.}
  \label{fig:karma-output}
\end{figure}

\makesection{Testy architektury serwera}{sec:server-tests}

Większość architektury serwera (punkty końcowe \textit{REST}) zostały przetestowane z użyciem narzędzia
\textit{Postman}. Jest to program umożliwiający sprawdzenie poprawności zapytań \textit{HTTP}. Jego specyfika działania
powoduje, że jest on również interaktywną dokumentacją do projektu. Umożliwia deklarowanie zmiennych, które mogą
przyjmować inne wartości dla różnych środowisk oraz wprowadza możliwość uwierzytelniania zasobów przy pomocy
\textit{JWT} co dodatkowo usprawnia testowanie. Same zapytania przechowywane są w folderach, wprowadzając porządek, co
jest istotne przy dużych projektach programistycznych. Na \imgref{fig:postman} widoczne jest okno programu
\textit{Postman} z przykładowym zapytaniem \textit{HTTP} i jego rezultatem.
%
\begin{figure}[H]
  \centering
  \safeincludepng{5_2/0_1_postman}{1}
  \caption{Okno programu \textit{Postman} z widocznym zapytaniem \textit{HTTP}.}
  \label{fig:postman}
\end{figure}

Dodatkowo do poszczególnych fragmentów kodu w mikroserwisach zostały napisane testy jednostkowe z wykorzystaniem
bibliotek \textit{Spring Boot Starter Test} oraz \textit{JUnit}. Na listingu \lisref{lis:java-test} widoczny jest
przykład testu jednostkowego sprawdzającego poprawność działania metody konwertującej nazwę \textit{Sfery} lub
użytkownika na inicjały wyświetlane w formie grafiki (domyślne zdjęcie profilowe).
%
\begin{lstlisting}[
  language=Java,
  label={lis:java-test},
  caption={Test jednostkowy metody \texttt{parseGuildNameInitials()}.}
]
@Test
void parseGuildNameInitials() {
  // given
  final String[] mockedNames = {
    "testguild123", "test test123", "GGD2"
  };
  final String[] expectedResults = { "t", "tt", "G" };

  // when
  final String[] actualResults = new String[expectedResults.length];
  for (int i = 0; i < mockedNames.length; i++) {
    actualResults[i] = new String(
      StringParser.parseGuildNameInitials(mockedNames[i])
    );
  }

  // then
  assertArrayEquals(expectedResults, actualResults);
}
\end{lstlisting}

Metoda ta (listing \lisref{lis:java-method}) rozdziela ciąg wejściowy względem delimitera (w tym przypadku znaku spacji)
a następnie pobiera pierwszy znak każdego rozdzielonego elementu. Jeśli ilość elementów jest mniejsza od dwóch, pobiera
jedynie pierwszy znak. Każdy test jednostkowy w aplikacji podzielony jest na 3 etapy. Są to odpowiednio \verb|given|,
\verb|when| i \verb|then|. \verb|Given| symbolizuje dane wejściowe oraz oczekiwany rezultat, \verb|when| jest
uruchomieniem funkcji testującej a \verb|then| sprawdzeniem wyników. Jeśli dane wyjściowe nie będą zgodnie z
oczekiwanymi, test zwróci błąd z widocznym stosem wywołań (\imgref{fig:junit-error}). Jeśli natomiast test zostanie
zakończony pomyślnie, zostanie zwrócona wartość 0 oznaczająca zakończenie operacji sukcesem.
%
\begin{lstlisting}[language=Java,label={lis:java-method},caption={Ciało metody \texttt{parseGuildNameInitials()}.}]
public static char[] parseGuildNameInitials(String guildName) {
  final String[] parts = guildName.split(StringUtils.SPACE);
  final char[] initials;
  if (parts.length > 1) {
    initials = new char[2];
    for (int i = 0; i < 2; i++) {
      initials[i] = parts[i].charAt(0);
    }
  } else {
    initials = new char[]{ parts[0].charAt(0) };
  }
  return initials;
}
\end{lstlisting}
%
\begin{figure}[H]
  \centering
  \safeincludepng{5_2/0_2_junit_error}{1}
  \caption{Wynik testu z widocznym błędem.}
  \label{fig:junit-error}
\end{figure}

\makesection{Napotkane problemy}{sec:issues}

\makesubsection{
  Przekazywanie stanu pomiędzy sesje Websocket w wielu instancjach mikroserwisu \textit{chat-service}
}{sub:ws-state-issue}

W protokole \textit{Websocket} w przeciwieństwie do \textit{HTTP} nie jest możliwym zaprojektowania komunikacji
bezstanowej z uwagi na sposób jego działania (ciągłość działania poprzez otwarty kanał komunikacyjny z serwerem). Z tego
powodu przy uruchomionych wielu instancjach mikroserwisu czatu, klient mógł się przyłączyć do sesji \textit{Websocket}
do której nie był przyłączony odbiorca, co skutkowało brakiem propagowania wiadomości kanału tekstowego. W celu
rozwiązania tego problemu skorzystano z usługi propagacji przy użyciu brokera wiadomości \textit{RabbitMQ}. W działaniu
przypomina oprogramowanie \textit{Apache Kafka}, lecz w przeciwieństwie do używania własnego protokoły komunikacji
opartego o \textit{TCP}, implementuje on interfejs \textit{JMS} oraz wspiera protokół \textit{STOMP}, co jest kluczową
zaletą z uwagi na wykorzystanie jego przy komunikacji z użyciem protokołu \textit{Websocket}. Sam \textit{RabbitMQ}
pośredniczy pomiędzy klientem \textit{Websocket} a serwerem tworząc pule połączeń oraz propagując wiadomości wysyłane
przez użytkownika do wszystkich aktywnych sesji w wielu instancjach mikroserwisu, a co za tym idzie aktywnych
użytkowników zainteresowanych odbiorem (na tym samym kanale tekstowym) \cite{bib:real-time-app-scaling}.
