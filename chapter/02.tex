\makechapter{Wykorzystane oprogramowanie}{chap:used-software}

\makesection{Wykorzystane oprogramowanie do rozwoju warstwy klienta}{sec:client-software}

W celu stworzenia warstwy klienta składającego się na główny projekt \textit{SPA} napisany we \textit{frameworku}
\textit{Angular} oraz podprojekt zawierający style biblioteki \textit{Tailwind} wraz z dodatkową konfiguracją
\textit{Webpack} skorzystano z programu \textit{Visual Studio Code}. Jest to lekki edytor kodu źródłowego dostępnego na
systemy \textit{Windows}, \textit{MacOS} oraz \textit{Linux}. Możliwe jest instalowanie do niego rozszerzeń
integrujących elementy innych języków oraz frameworków (takie jak kompilatory czy walidatory składni) \cite{bib:vscode}.
W powyższym projekcie skorzystano głównie z rozszerzeń ekosystemu \textit{Angular} oraz \textit{Tailwind}
umożliwiających sprawdzanie poprawności kodu przed jego interpretacją co znacznie usprawniło i przyspieszyło rozwój
systemu.

\makesection{Wykorzystane oprogramowanie do rozwoju warstwy serwera}{sec:server-software}

W celu stworzenia warstwy serwerowej wykorzystano programy firmy \textit{JetBrains}. Wśród nich znajduje się flagowy
produkt tej firmy — \textit{Intellij Idea}. Jest to program typu \textit{IDE} stosowany do pisania, testowania oraz
kompilowania programów uruchamianych na \textit{JVM} (napisanych w \textit{Javie}, ale również w innych językach takich
jak \textit{Groovy} czy \textit{Kotlin}). Podobnie jak \textit{Visual Studio Code} umożliwia instalowanie wtyczek
dodających nowe funkcje \cite{bib:intellij-idea}. W projekcie skorzystano ze wtyczki \textit{Apache Kafka} pozwalającej
na obserwację napływających danych do strumieni co znacznie usprawniło walidację poprawnego przepływu danych między
serwisami.

Dodatkowo skorzystano z programu \textit{DataGrip} (również od firmy \textit{JetBrains}) będącym kompleksowym
środowiskiem bazodanowym. Aplikacja ta umożliwia podpięcie wielu różnych baz danych (różnych typów o różnych
specyfikacjach) i sprawnie nimi zarządzanie \cite{bib:intellij-datagrip}. Funkcja ta była dużym atutem w trakcie rozwoju
aplikacji ze względu na wykorzystaną architekturę, która narzuca stosowanie odrębnej bazy danych dla każdego
mikroserwisu.

\makesection{System kontroli wersji \textit{GIT}}{sec:git}

W rozwoju całego systemu korzystano z systemu kontroli wersji \textit{GIT}. System kontroli wersji to oprogramowanie
zapisujące historię edycji plików z kodem źródłowym. Sam system \textit{GIT} został stworzony przez Linusa Torvaldsa,
twórcę jądra systemu operacyjnego \textit{Linux} \cite{bib:git-history}. Dodatkowo system taki jest rozproszony, dzięki
czemu umożliwia pracę na wielu różnych gałęziach nad różnymi funkcjami systemu w tym samym czasie
\cite{bib:git-overview}. Stał się on bardzo przydatny podczas dodawania nowych funkcjonalności do dosyć złożonego
architektoniczne projektu przy pracy z wykorzystaniem dwóch komputerów.

\makesection{System konteneryzacji \textit{Docker}}{sec:docker}

W celu ułatwienia przyszłego wdrożenia systemu w środowiskach produkcyjnych skorzystano z systemu konteneryzacji
\textit{Docker}. Sam system \textit{Docker} powstał jako szybsza i wydajniejsza alternatywa dla tradycyjnych systemów
wirtualizacji. Główne dwa elementy w systemie \textit{Docker} to obraz i kontener. Obraz zawiera cały kod źródłowy i
wszystkie pliki niezbędne do uruchomienia aplikacji. Kontener zaś to odseparowane środowisko (tzw.
\textit{piaskownica}), które umożliwia uruchomienie wcześniej przygotowanego obrazu jako niezależny proces na komputerze
hosta \cite{bib:docker-overview}. Umożliwia to przede wszystkim uruchamianie wielu instancji tej samej aplikacji z taką
samą konfiguracją niezależnie od zastosowanej maszyny, co jest częstą praktyką w systemach rozproszonych opartych o
architekturę mikroserwisową w celu równoważenia obciążenia.
