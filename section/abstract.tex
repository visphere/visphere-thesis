\subsubsection*{Tytuł pracy}
\Title

\subsubsection*{Streszczenie}
Projekt ma na celu przedstawienie możliwości wymiany danych między aplikacjami w środowisku rozproszonym przy użyciu
Apache Kafka oraz komunikacji z klientem z wykorzystaniem protokołu Websocket na przykładzie prostej aplikacji wymiany
wiadomości w czasie rzeczywistym. Wykonanie pracy polegało na stworzeniu dwóch osobnych warstw: wizualnej oraz
serwerowej. Architektura serwerowa została rozdzielona na kilka mikrousług odpowiadających za konkretne cele biznesowe
aplikacji (uwierzytelnianie, obsługa czatu, przetwarzanie plików itp.). Każdy mikroserwis stanowi osobny serwer
aplikacji z własną bazą danych. Komunikacja między mikroserwisami została zapewniona przez architekturę sterowaną
zdarzeniami i system Apache Kafka. Sama komunikacja z aplikacją klienta została zrealizowana przy użyciu interfejsu
REST, protokołu Websocket oraz zaimplementowanego serwera brzegowego. Wykonany system umożliwia kilka operacji. Są to
m.in. utworzenie konta, zarządzanie pokojami do rozmów, kanałów tekstowych oraz wysyłanie wiadomości z opcjonalnymi
załącznikami w formie plików. Stworzony system mimo swojej zewnętrznej prostoty zawiera wiele zaawansowanych konceptów z
zakresu architektury mikrousługowej. Rozwiązują one problemy aplikacji projektowanych na systemy rozproszone które nie
występują (lub występują w nieznacznym stopniu) w aplikacjach monolitycznych.

\subsubsection*{Słowa kluczowe}
system czatu, mikroserwisy, architektura sterowana zdarzeniami, apache kafka, websockety, apache cassandra, kontenery
docker

\newpage

\begin{otherlanguage}{british}

  \subsubsection*{Thesis title}
  \TitleAlt

  \subsubsection*{Abstract}
  The following project was designed to demonstrate the possibility of data exchange between applications in distributed
  environments using the Apache Kafka platform and communication with client using Websocket protocol on the example of
  a simple real-time messaging application. Main goal of this project involved creating two separate layers: visual
  layer and server layer. The server architecture was divided into several microservices responsible for specific
  business purposes in the application (authentication, chat, file processing, etc.). Each microservice is a separate
  application server with its own database. Communication between microservices was provided by the event-driven
  architecture and an Apache Kafka software. Communication between client application and server layer was implemented
  using a REST interface, Websocket protocol and an API gateway server. This system allows us to perform several
  operations. These include creating an account, managing chat rooms, text channels and sending messages with
  attachments. Despite its external simplicity, the created system incorporates many advanced concepts in the field of
  microservices architecture. These concepts address issues in applications designed for distributed systems that are
  not present (or are present to a lesser extent) in monolithic applications.

  \subsubsection*{Keywords}
  instant messenger system, microservices, event-driven architecture, apache kafka, websocket, apache cassandra, docker
  containers

\end{otherlanguage}
